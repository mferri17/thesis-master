

%\setcounter{chapter}{0}
\chapter{Test features}
\label{chap:test}

\glsreset{ai}
\glsreset{dl}

In this work we consider cooperative multi-agent scenarios, in which multiple 
robots collaborate, and possibly communicate, to achieve a common goal 
\cite[][]{ismail2018survey}. Let'try this \footnote{this is a footnote}.
Homogeneous Multi-Agent Systems (MAS) are composed of $N$ interacting 
agents, which have the same physical structure and observation capabilities, so 
they can be considered to be interchangeable and cooperate to solve a given 
task \cite[][]{stone2000multiagent, vsovsic2016inverse}.
This system is characterised by a state $S$ — which can be decomposed in sets of 
local states for each agent and the set of possible observations $O$ for each agent 
— obtained through sensors and the set of possible actions  $A$ for each agent.

\begin{figure}[!htb]
	\begin{center}
		\begin{subfigure}[h]{0.49\textwidth}
			\centering
			\includegraphics[width=\textwidth]{"contents/images/20210110_172800 - origtest11030(len11030)_fps30_w2"}
		\end{subfigure}
		\hfill
		\begin{subfigure}[h]{0.49\textwidth}
			\centering
			\includegraphics[width=\textwidth]{"contents/images/20210110_172810 - origtest11030(len11030)_fps30_w5"}
		\end{subfigure}
	\end{center}
	\vspace{-0.5cm}
	\caption[Visualisation of the simulation of the first task.]{Visualisation of the 
		initial and final configurations obtained simulating the first task.}
	\label{fig:task1}
	%	\vspace{-0.5cm}
\end{figure}

\lipsum[2]

\bigskip

\begin{figure}[!htb]
	\centering
	\includegraphics[width=.55\textwidth]{"contents/images/20210110_172810 - origtest11030(len11030)_fps30_w5"}
	\caption[Non-holonomic differential drive mobile robot.]{Configuration of a 
		non-holonomic differential drive mobile robot 
		\cite[][]{shojaei2011adaptive}.}
	\label{fig:differentialdrive}
\end{figure}

\lipsum[2]

\bigskip

\begin{python}
c_left, c_right = get_received_communication(state)

if N % 2 == 1:  # if the number of robots is odd

# Case 1: no communication received from left
if c_left == 0:
if c_right > N // 2:
# the agent is in the first half of the row, so its colour is blue
message = c_right - 1
colour = 1
elif c_right == N // 2:
# the agent is the central one, so its colour is blue
message = c_right + 1
colour = 1
\end{python}
\begin{lstlisting}[frame=none,caption={Protocol used by the manual controller 
to decide, for each robot, the message to transmit and the colour.}, 
label=lst:manualtask2]
\end{lstlisting}


\paragraph*{\texttt{Numpy}}
\lipsum[1]


\paragraph*{\texttt{Keras}}
\lipsum[1]