

%\setcounter{chapter}{0}
\chapter{Latex features}
\label{chap:test}

\glsreset{ai}
\glsreset{dl}

\begin{verbatim}
ACCURACY    10-FOLD CROSS VALIDATION:    0.8290 (std dev 0.005217)
PRECISION   10-FOLD CROSS VALIDATION:    0.8371 (std dev 0.009706)
RECALL      10-FOLD CROSS VALIDATION:    0.8176 (std dev 0.008412)
F1          10-FOLD CROSS VALIDATION:    0.8273 (std dev 0.004608)
\end{verbatim}

\begin{itemize}
	\item prodotti e utenti nella stessa rete, collegati attraverso le recensioni
	\item rete di soli prodotti, collegati tramite similarità
	\item rete di soli utenti, collegati tramite similarità
\end{itemize}

\begin{figure}[!htb]
	\begin{center}
		\begin{subfigure}[h]{0.30\textwidth}
			\centering
			\includegraphics[width=\textwidth]{"contents/images/20210110_172800 - origtest11030(len11030)_fps30_w2"}
		\end{subfigure}
		\hfill
		\begin{subfigure}[h]{0.68\textwidth}
			\centering
			\includegraphics[width=\textwidth]{"contents/images/20210110_172810 - origtest11030(len11030)_fps30_w5"}
		\end{subfigure}
	\end{center}
	\vspace{-0.5cm}
	\caption[list of figures caption]{current caption}
	\label{fig:task1}
	%	\vspace{-0.5cm}
\end{figure}

\begin{figure}[!htb]
\centering
\includegraphics[width=.55\textwidth]{"contents/images/20210110_172810 - origtest11030(len11030)_fps30_w5"}
\caption[list of figures caption]{current caption \cite[][]{shojaei2011adaptive}.}
\label{fig:differentialdrive}
\end{figure}

\lipsum[2]

\bigskip

\begin{python}
def tfdata_generator(files, input_size, batch_size, backgrounds, 
					 bg_smoothmask, aug_prob=0, noises=[],
					 prefetch=True, parallelize=True, 
					 deterministic=False, cache=False, repeat=1):

	map_parallel = tf.data.experimental.AUTOTUNE if parallelize else None
	backgrounds = tf.convert_to_tensor(backgrounds) # saves time during training
	noises = tf.convert_to_tensor(noises) # saves time during training

	gen = tf.data.Dataset.from_tensor_slices(files)
	gen = gen.map(lambda filename: map_parse_input(filename, input_size), map_parallel, deterministic)
\end{python}

\begin{lstlisting}[frame=none,caption={Protocol used by the manual controller 
to decide, for each robot, the message to transmit and the colour.}, 
label=lst:manualtask2]
\end{lstlisting}


\begin{table}[H]
	\caption{Schema originale del dataset}\label{tab:df-schema}
	\centering
	\begin{tabular}{|l|l|l|}
		\hline
		Campo & Descrizione \\
		\hline
		reviewerID & ID utente \\
		reviewerName & Nome utente \\
		asin & ID prodotto \\
		\hline
	\end{tabular}
\end{table}


\paragraph*{\texttt{Numpy}}
Good library


\paragraph*{\texttt{Keras}}
Very good for ML