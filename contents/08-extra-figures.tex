
\chapter{Extra Figures}
\label{chap:extra-figures}
%\let\clearpage\relax

This appendix lists a bunch of images not inserted in the main chapters to keep some section shorter and enhance general readability. Following figures are not crucial for the understanding of our work, but they still add minor interesting details.




\section{FrontalNet}
\label{sec:extra-frontalnet}

\begin{figure}[!h]
	\centering
	\includegraphics[width=0.99\textwidth]{"contents/images/03-frontalnet-1"}
	\caption[Models by \cite{mantegazza2019visionbased}: mediated, end-to-end, learned controlled]{Models by \cite{mantegazza2019visionbased}: mediated, end-to-end, learned controlled}
	\label{fig:frontalnet-architecture-paper3approaches}
\end{figure}

Please note that following architecture is for classification purposes presented in Section \ref{subsec:gradcam-regrtoclass}. Standard model outputs have shapes \texttt{(?, 1)} instead.

\begin{figure}[!h]
	\centering
	\includegraphics[width=0.9\textwidth]{"contents/images/03-frontalnet-3A"}
	\caption[FrontalNet complete architecture (part 1)]{FrontalNet complete architecture (part 1, from input to layer 18)}
	\label{fig:frontalnet-architecture-3a}
\end{figure}

\begin{figure}[!h]
	\centering
	\includegraphics[width=0.9\textwidth]{"contents/images/03-frontalnet-3B"}
	\caption[FrontalNet complete architecture (part 2)]{FrontalNet complete architecture (part 2, from layer 18 to outputs)}
	\label{fig:frontalnet-architecture-3b}
\end{figure}

\clearpage

Figure \ref{fig:frontalnet-trajectories} presents qualitative results from \cite{mantegazza2019visionbased}. It shows the trajectories followed by the drone during five consecutive flights in the drone arena. The quadrotor has to face a user initially rotated by 90 degrees. Although the paths (obtained by two distinct models A1 and A2) are sometimes different from what designed by the omniscient controller (the ground truth), they are still reasonable, and flying capabilities inside the drone arena seem very promising.

\begin{figure}[!htb]
	\centering
	\includegraphics[width=0.8\textwidth]{"contents/images/03-frontalnet-trajectories"}
	\caption[FrontalNet trajectories for positioning in front of the user initially rotated by 90 degrees \cite{mantegazza2019visionbased}]{FrontalNet trajectories for positioning in front of the user initially rotated by 90 degrees \cite{mantegazza2019visionbased}}
	\label{fig:frontalnet-trajectories}
\end{figure}

\clearpage


\section{Grad-CAM}
\label{sec:extra-gradcam}

This sections reports \gls{gradcam} applications appropriately divided into variables and classes, in contrast with the single-image approach followed in Section \ref{subsec:gradcam-results}.

\begin{figure}[!h]
	\centering
	\includegraphics[width=0.8\textwidth]{"contents/images/gradcam/apx-gradcam-1"}
	\caption[Full \gls{gradcam}: two people in the frame]{Full \gls{gradcam}: two people in the frame}
	\label{fig:apx-gradcam-1}
\end{figure}

\begin{figure}[!h]
\centering
\includegraphics[width=0.8\textwidth]{"contents/images/gradcam/apx-gradcam-2"}
\caption[]{Full \gls{gradcam}: two people in the frame}
\label{fig:apx-gradcam-2}
\end{figure}

\begin{figure}[!h]
\centering
\includegraphics[width=0.8\textwidth]{"contents/images/gradcam/apx-gradcam-3"}
\caption[Full \gls{gradcam}: model is attracted by curtains]{Full \gls{gradcam}: model is attracted by curtains}
\label{fig:apx-gradcam-3}
\end{figure}

\begin{figure}[!h]
\centering
\includegraphics[width=0.8\textwidth]{"contents/images/gradcam/apx-gradcam-4"}
\caption[]{Full \gls{gradcam}: model is attracted by curtains}
\label{fig:apx-gradcam-4}
\end{figure}

\begin{figure}[!h]
\centering
\includegraphics[width=0.8\textwidth]{"contents/images/gradcam/apx-gradcam-5"}
\caption[Full \gls{gradcam}: model is attracted by background objects]{Full \gls{gradcam}: model is attracted by background objects}
\label{fig:apx-gradcam-5}
\end{figure}

\begin{figure}[!h]
\centering
\includegraphics[width=0.8\textwidth]{"contents/images/gradcam/apx-gradcam-6"}
\caption[]{Full \gls{gradcam}: model is attracted by background objects}
\label{fig:apx-gradcam-6}
\end{figure}