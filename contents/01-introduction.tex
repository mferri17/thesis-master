\chapter{Introduction}
%\addcontentsline{toc}{chapter}{Introduction}
%\markboth{}{}
\label{chap:intro}

\glsreset{ai}
\glsreset{dl}


\lipsum[1]




\section{Objective}
\label{sec:objective}

\lipsum[2-5]




\section{Outline}
\label{sec:outline}


The thesis is composed of X chapters, whose main points are presented as follows:
\begin{itemize}
	\item Chapter \ref{chap:} summarises the previous research 
	on the topic, evaluating the approaches adopted by the authors;
	
	\item Chapter \ref{chap:} provides the background knowledge 
	needed to properly understand the research contents;
	
	\item Chapter \ref{chap:} presents the tools used for the data collection 
	and all the additional frameworks we relied on;
		
	\item Chapter \ref{chap:} thoroughly illustrates the methodology used, 
	their benefits and limitations, also including descriptions of the kind of data 
	used and how they are collected;

	\item Chapter \ref{chap:} explores the analysis conducted and 
	shows evaluation results;	
	
	\item The \ref{chap:} addresses the results of the 
	experiments, concludes the thesis by discussing the implications of our findings, 
	possible improvements and outlines future works.
\end{itemize}





