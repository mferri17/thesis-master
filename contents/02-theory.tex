\chapter{Theoretical Foundation}
\label{chap:theory}

%\lipsum[1]



\section{Robotics}
\label{sec:robotics}

%\subsection{Pose}
%\label{subsec:robot-pose}
%
%\subsection{Control Theory}
%\label{subsec:robot-control}




\section{Machine Learning}
\label{sec:machine-learning}

%\subsection{Supervised Learning}
%\label{subsec:supervised-learning}
%
%\subsection{Convolutional Neural Networks}
%\label{subsec:cnn}
%
%\subsection{Residual Neural Networks}
%\label{subsec:resnet}




\section{Human-drone Interaction}
\label{sec:human-drone-interaction}

A good variety of research can be found on human-robot interaction and a lot is yet to come. In the field, drones represent a specific segment due to their ability to freely move in the 3D space, opening access to new use cases while representing a real challenge for professionals and researchers.

In this section, we firstly present a general overview on the topic, then we focus on related works by \gls{idsia}.



\subsection{The State of the Art of Human–Drone Interaction}
\label{subsec:human-drone-sota}

An article published in Nov 2019 for IEEE Access (\cite{human-drone-sota}), explores literature and state of the art for human-drone interaction. Drones range from small toy-grade remote-controlled aircraft to fully-autonomous systems capable of decision-making through a large variety of sensors. Their usage grew a lot in the last years and it is expected to keep growing, thanks to decreasing costs and powerful features they can provide both for personal, commercial, and social usage.

\gls{faa} expects that total drone registrations will increase by more than 60\% between 2018 and 2022 with a particular increment in the commercial sector rather than the hobbyist one, even though the latter still counts the largest number of units. Moreover, \gls{faa} reports that almost half of drones usage is for aerial photography (48\%), followed by industrial inspection (28\%) and agriculture (17\%). Accordingly to \cite{human-drone-sota}, drones will become ubiquitous to society and, in the next decade, they will be extensively used in advertising, shipping, sports, emergency, and many other fields for augmenting human capabilities.

The main concerns about drones today regard safety issues caused by propellers and limited flight times, usually no longer than 30 minutes due to limited battery capacity. Research in the sector of human-drone interaction mainly focuses on their control (through gestures, voice, or custom interfaces), communication between the user and the drone itself (in terms of acknowledgment and intents), perception of users' safety during flight, and innovative use cases. 



\subsection{Vision-based Control of a Quadrotor in User Proximity}
\label{subsec:sota-dario}

Our work is built upon the original master thesis and paper from Dario Mantegazza named \textit{Vision-based Control of a Quadrotor in User Proximity: Mediated vs End-to-End Learning Approaches} from Dario Mantegazza, developed at \gls{idsia} between 2018 and 2019 (\cite{mantegazza2018thesis}, \cite{mantegazza2019visionbased}).

\medskip

In his thesis, the author proposes a machine learning model for teaching a drone to interact with a person by continuously flying to face the user frontally. The problem is approached as a reactive control procedure and addressed with supervised learning, thus provides an interesting starting point for many other robotics applications. 

A considerable amount of flights is recorded for building the training data by programmatically flying the drone in front of a person, controlling it through an omniscient controller that knows both the drone's and user's pose. Images produced by the front-facing camera of the drone are used as input for a custom-designed \gls{resnet} architecture to infer the relative user's position \gls{wrt} the drone. Practically, the neural network performs a regression on the four variables that form the user's pose (X, Y, Z, YAW) and learns to predict their values by using spacial information contained in the input images. 

In the paper, the author also makes a comparison between the mediated approach described above and another end-to-end approach that directly learns control signals\footnote{desired pitch, roll, yaw, and vertical velocity} for the drone, instead of the user's pose. Another experiment also considers a learned controller. All the solutions provide similar results, but the former can be adapted to other tasks by simply designing a custom controller, providing a more transparent and analyzable solution.

\medskip

Even though this kind of problems on human recognition and pose estimation could be faced with more advanced deep learning algorithms, making a simple regression on four variables allows the network to be fairly small, so that the prediction task is light, fast to execute, and possibly portable on low-end devices.

\bigskip

Our entire project makes use of the original network architecture and dataset defined in \cite{mantegazza2019visionbased}, respectively explained in section \ref{subsec:frontalnet-architecture} and \ref{sec:dataset}, For a better understanding, the original code repository\footnote{\url{https://github.com/idsia-robotics/proximity-quadrotor-learning}} and a descriptive video are available\footnote{\url{https://drive.switch.ch/index.php/s/MlEDrsuHcSl5Aw5}}. 
For enhancing readability of this thesis, having no official name, the custom \gls{resnet} architecture proposed by the author will be simply called \textit{FrontalNet}.


\subsubsection{FrontalNet Architecture}
\label{subsec:frontalnet-architecture}

The network is composed of a total of 1'332'484 trainable parameters, accepts a single $60 \times 108$ pixels image in input, and outputs 4 regression variables that correspond to the user's pose coordinates. Each \gls{resnet} block is provided with batch normalization, ReLU activations (for info, \cite{act-relu}) are used for all layers except for the output neurons, which are associated with a linear activation function (for info, \cite{act-linear}). 

Figure \ref{fig:frontalnet-architecture-1} provides an illustration of the mediated architecture we consider for this thesis\footnote{take a look at image \ref{fig:frontalnet-architecture-paper3approaches} for details about the architectures related to the other approaches}. A complete list of all layers is also available in figures \ref{fig:frontalnet-architecture-3a} and \ref{fig:frontalnet-architecture-3a} of the appendix. 

\begin{figure}[!h]
	\centering
	\includegraphics[width=0.3\textwidth]{"contents/images/03-frontalnet-1A"}
	\caption[Schematic FrontalNet architecture]{Schematic FrontalNet architecture (\cite{mantegazza2019visionbased})}
	\label{fig:frontalnet-architecture-1}
\end{figure}


\subsubsection{FrontalNet Performance}
\label{subsec:frontalnet-performance}

FrontalNet is trained using the \gls{mae} loss function with the \gls{adam} optimizer (for info, \cite{kingma2014adam}) and a base learning rate of 0.001, progressively reduced on validation loss plateaus that last more than 5 epochs. A maximum of 200 epochs are run in total, but with an early stopping policy with patience of 10 epochs on the validation loss. 

Performance is evaluated both quantitatively and qualitatively on the end-to-end model, rather than the mediated one considered for our work. However, as explained before, both approaches obtain similar results. We can accordingly consider the following evaluation to be valid for the mediated approach too.

\medskip 

\glsreset{r2}
For quantitative evaluation, the chosen metric is \gls{r2}\footnote{\gls{r2} interpretation will be explained in the evaluation chapter \ref{chap:evaluation}}, which has an interval of $[-\inf, 1]$, where 1 represents the optimality. 

The author also conducts an experiment about the minimum cardinality of the dataset for obtaining acceptable performance. Results are available in figure \ref{fig:frontalnet-r2}, directly taken from the paper. As shown, decent performance requires at least 5'000 samples and keep improving as their number increases.

Specifically, predictions seem more accurate for variables Z and W with an \gls{r2} score of 0.82 and 0.88, respectively. Different the findings for X and Y which only reach an \gls{r2} of 0.59 and 0.57, respectively.

\begin{figure}[!htb]
	\centering
	\includegraphics[width=0.8\textwidth]{"contents/images/03-frontalnet-r2"}
	\caption[FrontalNet \gls{r2} results (\cite{mantegazza2019visionbased})]{FrontalNet \gls{r2} results (\cite{mantegazza2019visionbased}). A1, A2 and A3 in the chart stands for different models, but they anyway achieve almost the same results.}
	\label{fig:frontalnet-r2}
\end{figure}

\medskip

The previous considerations on the variables are confirmed by the qualitative evaluation, obtained by comparing ground truth and predictions during a short simulation. Figure \ref{fig:frontalnet-gt-pred} shows that X and Y predictions are considerably worse than results achieved by Z and W when compared with the ground truth.

\begin{figure}[!htb]
	\centering
	\includegraphics[width=0.8\textwidth]{"contents/images/03-frontalnet-gt-pred-2"}
	\caption[FrontalNet GT vs prediction results (\cite{mantegazza2019visionbased})]{FrontalNet GT vs prediction results (\cite{mantegazza2019visionbased}). A1, A2 and A3 in the chart stands for different models, but they anyway achieve almost the same results.}
	\label{fig:frontalnet-gt-pred}
\end{figure}


\subsection{Embedded Implementation of Controller for Nano-Drones}
\label{subsec:sota-nicky}

Autonomous navigation is an important and well-known area of research in robotics, which usually requires accomplishing complex and computationally-expensive tasks such as localization, mapping and path planning. Recent studies have started to approach autonomous driving through imitation learning\cite{imitation-learning}, where neural networks learn by imitating human behavior in specific tasks. 

\medskip

In 2018, researchers at the UZH University of Zürich have demonstrated that \gls{resnet}s are able to provide satisfactory performance in the field (\cite{Loquercio_2018}). They developed DroNet, a forked \gls{cnn} that predicts, from a single gray-scale image, a steering angle and a collision probability. In other words, the model learns to steer and avoid obstacles from forward-looking videos recorded by cars and bikes while driving in real contexts. In this case, both the prediction and controller tasks were powered off-drone on a dedicated computer, remotely connected through WiFi. 

A year later, ETH Zürich was able to develop PULP-DroNet, porting the \gls{cnn} on the Crazyflie\footnote{\url{https://www.bitcraze.io/products/crazyflie-2-1/}}, a nano-drone with a size of only 3 $3 \times 3$ centimeters for a weight of 27 grams. They propose a general methodology for deploying on-board deep learning algorithms for ultra-low-power devices \cite{palossi2019pulpdronetIoTJ}, without any need for an external laptop to run the software.

\medskip

Inspired by PULP-DroNet, \gls{idsia} adapted its FrontalNet to work on-board the Crazyflie with excellent results (\cite{zimmerman2020thesis}). The nano-drone is able to achieve good quantitative and qualitative performance, regardless of any problem deriving from working with such low-end devices. The main challenges are represented by low computational power, energy consumption management, and low-fidelity camera with no video stabilization\footnote{Himax HM01B0 camera, theoretically able to produce $320 \times 320$ \gls{mp} images at 60 FPS. However, the frame rate is incredibly reduced during data collection due to some platform limitation in image transfering.}.




\section{Network Interpretability}
\label{sec:network-interpretability}

\glsreset{ml}
\glsreset{nn}

Deep \gls{nn}s learn abstract representations for finding a logical mapping between their input and output, determined by well-defined mathematical computations that involve the input itself and the progressively learned network parameters. Inspired by biological brains, this approach seems to be incredibly effective on a huge variety of tasks.

\medskip 

However, unlike other \gls{ml} techniques, \gls{nn} are known to produce "black-box" models, particularly hard to understand even from domain experts. Their reasoning and comprehension are intrinsic in the network parameters, which are nothing but numbers.

When working with real-world problems, it is extremely important to be able to explain what a \gls{ml} model is actually understanding. This builds trust in algorithms and makes sure there are no undesirable biases in the models, which could raise serious problems, especially in critical fields such as medicine and law.

\medskip

\gls{xai} is the field of study which tries to make \gls{ml} results, and their underlying basis for decision-making, properly understandable to humans (\cite{xai-wiki}). For \gls{cnn}s, researchers have developed many techniques for understanding what a \gls{nn} actually care of when producing an output based on an input image. 

Main efforts regard feature visualization and attribution, but recent advanced studies have also shown how these methods can be used altogether (\cite{olah2018the}). This section briefly explains these two major areas for \gls{cnn} interpretability, with a particular focus on spatial attribution, the chosen methodology for our work.



\subsection{Feature Visualization}

Sources: \cite{olah2017feature}



\subsection{Spatial Attribution with GradCAM}
\label{subsec:gradcam-theory}

Sources: \cite{Selvaraju_2019}, \cite{gradcam_medium}

\begin{figure}[!htb]
	\centering
	\includegraphics[width=0.9\textwidth]{"contents/images/gradcam/02-gradcam-schema"}
	\caption[\gls{gradcam} schematic functioning]{\gls{gradcam} schematic functioning (\cite{Selvaraju_2019})}
	\label{fig:gradcam-schema}
\end{figure}

\begin{figure}[!htb]
\centering
\includegraphics[width=0.9\textwidth]{"contents/images/gradcam/02-gradcam-catdog"}
\caption[\gls{gradcam} example on dog-cat classification]{\gls{gradcam} example on dog-cat classification (\cite{Selvaraju_2019})}
\label{fig:gradcam-catdog}
\end{figure}




\section{Network Generalization}
\label{sec:network-generalization}

Asd



\subsection{Data Augmentation}
\label{subsec:data-augmentation}

\textbf{TODO}

Random Erasing Data Augmentation: \cite{zhong2017random}

Previous works on images:  \cite{yue2019domain}, \cite{Takahashi_2020}, \cite{xie2020unsupervised}

AutoAugment: \cite{cubuk2019autoaugment}
Learning Data Augmentation Strategies for Object Detection: \cite{zoph2019learning}



\subsection{Domain Randomization}
\label{subsec:domain-randomization}

\textbf{TODO}

Summary from \cite{mehta2019active}

See \cite{weng2019DR_explanation}, \cite{tobin2017domain}
 
"Domain randomization is a popular technique for improving domain transfer, often used in a zero-shot setting when the target domain is unknown or cannot easily be used for training. In this work, we empirically examine the effects of domain randomization on agent generalization. Our experiments show that domain randomization may lead to suboptimal, high-variance policies, which we attribute to the uniform sampling of environment parameters."

\medskip

Imitation learning through simulation is recently becoming an interesting and successful approach for both reinforcement learning (\cite{imitation_learning_survey}, \cite{imitation_learning_3d_navigation}) object recognition (\cite{tobin2017domain}, \cite{weng2019DR}).

Robot and environment can be replicated through a dedicated simulator such as Gazebo\footnote{\url{http://gazebosim.org/}}, often used in robotics with \gls{ros}\footnote{\url{https://www.ros.org/}} due to its straightforward integration, or even with general-purpose graphic engines. Unreal Engine\footnote{\url{https://www.unrealengine.com/en-US/}} Unity\footnote{\url{https://unity.com/}} are well-known simulators designed for game-development, but recently used for \gls{vr} and \gls{ar} applications. They give developers unlimited possibilities, carefully supported by solid physics engines and active communities.

\medskip

Given the difficulty of collecting data for our task, exploring the possibility of simulating the entire scenario in a 3D virtual-world is intriguing, especially to replace the need for a complex \gls{mocap} system. Integrating odometry support, drone and people can be thoroughly modeled to act as in the real world, with similar movements and sensing capabilities, in order to collect the data very efficiently. The virtual simulation gives both the opportunity of reproducing real indoor/outdoor scenes, but also randomizing the background with artificially generated textures.

Even though the approach appears to obtain sub-optimal results, a complete and adaptable implementation requires a lot of effort, yet unlocking a huge number of possibilities. Considering the consistent amount of fine details to consider and issues that can arise during the development of such simulators, we opt instead to work with an easier generalization pipeline, that mostly concerns machine learning only.




\section{Human Detection and Segmentation}
\label{sec:sota-humandetseg}

\textbf{TODO}



\subsection{Chroma key}

A known approach for implementing background replacement is the chroma key. Widely used in entertainment, it is a technique that makes use of colors in images and videos for splitting between actual content and background. Usually, a chroma key is achieved through a blue or green screen placed behind the subject, making sure that such color is not present in the foreground image. Then, a post-production software takes care of creating the appropriate mask which separates the two parts and enables background replacement.

Although this technique is particularly popular in many fields, placing several green screens into the drone arena is not the easiest task since it requires a lot of material and physical work to set up the proper environment, with possible issues related to the room composition or its illumination.

An experiment in this direction has been conducted during development of \cite{mantegazza2019visionbased}, by placing a green screen on a portion of the arena walls. Masking results were actually satisfying, but the limitations of such a small green screen are huge both in terms of user's movements and background coverage. In fact, figure \ref{fig:greenscreen} displays the setup, which reveals a lot of classic background still appearing in the images.

In addition, even with the capability of building a well-designed chroma key environment, the solution would be highly dependent on the geographical location of the setup. On the contrary, software-only approaches would be much more portable and reusable together with any other motion capture system in the world.

\begin{figure}[!h]
	\begin{center}
		\begin{subfigure}[h]{0.24\textwidth}
			\centering
			\includegraphics[width=1\textwidth]{"contents/images/04-greenscreen-1"}
		\end{subfigure}
		\hfill
		\begin{subfigure}[h]{0.24\textwidth}
			\centering
			\includegraphics[width=1\textwidth]{"contents/images/04-greenscreen-2"}
		\end{subfigure}
		\hfill
		\begin{subfigure}[h]{0.24\textwidth}
			\centering
			\includegraphics[width=1\textwidth]{"contents/images/04-greenscreen-3"}
		\end{subfigure}
		\hfill
		\begin{subfigure}[h]{0.24\textwidth}
			\centering
			\includegraphics[width=1\textwidth]{"contents/images/04-greenscreen-4"}
		\end{subfigure}
	\end{center}
	\vspace{-0.5cm}
	\caption[Experimental green screen setup in the drone arena]{Experimental green screen setup in the drone arena}
	\label{fig:greenscreen}
\end{figure}



\subsection{Mask R-CNN}
\label{subsec:sota-maskrcnn}

\textbf{TODO}

MaskRCNN: \cite{he2018mask}, \cite{maskrcnn_explanation}, \cite{maskrcnn_arcgis}





