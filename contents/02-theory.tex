\chapter{Theoretical Foundation}
\label{chap:theory}

%\lipsum[1]



\section{Robotics}
\label{sec:robotics}

%\subsection{Pose}
%\label{subsec:robot-pose}
%
%\subsection{Control Theory}
%\label{subsec:robot-control}




\section{Machine Learning}
\label{sec:machine-learning}

%\subsection{Supervised Learning}
%\label{subsec:supervised-learning}
%
%\subsection{Convolutional Neural Networks}
%\label{subsec:cnn}
%
%\subsection{Residual Neural Networks}
%\label{subsec:resnet}




\section{Human-drone Interaction}
\label{sec:human-drone-interaction}

A good variety of research can be found on human-robot interaction and a lot is yet to come. In the field, drones represent a specific segment due to their ability of freely move in the 3D space, opening the access to new use cases while representing a real challenge for professionals and researchers.

In this section we firstly present a general overview on the topic, then we focus on related work by \gls{idsia}.



\subsection{The State of the Art of Human–Drone Interaction}
\label{subsec:human-drone-sota}

A recent article, published in Nov 2019 for IEEE Access (\cite{human-drone-sota}), explores literature and state of the art for human-drone interaction. Drones range from small toy-grade remote-controlled aircraft to fully-autonomous systems capable of decision-making through a large variety of sensors. Their usage grew a lot in the last years and it is expected to keep growing, thanks to decreasing costs and powerful features they can provide both for personal, commercial and social usage.

\gls{faa} expects that total drone registrations will increase by more than 60\% between 2018 and 2022 with a particular increment in the commercial sector rather than the hobbyist one, even though the latter still counts the largest number of units. Moreover, \gls{faa} reports that almost half of the drones usage is for aerial photography (48\%), followed by industrial inspection (28\%) and agriculture (17\%). Accordingly to \cite{human-drone-sota}, drones will become ubiquitous to society, and in the next decade they will be extensively used in advertising, shipping, sports, emergency, and many other fields for augmenting human capabilities.

Main concerns about drones today regards safety issues caused by propellers and limited flight times, usually no longer than 30 minutes due to limited battery capacity. Research in the sector of human-drone interaction mainly focus on their control (through gestures, voice or custom interfaces), communication between the user and the drone itself (in terms of acknowledgment and intents), perception of users' safety during flight, and innovative use cases. 



\subsection{Vision-based Control of a Quadrotor in User Proximity}
\label{subsec:sota-dario}

Our work is built upon the original master thesis (\cite{mantegazza2018thesis}) and paper (\cite{mantegazza2019visionbased}) \textit{Vision-based Control of a Quadrotor in User Proximity: Mediated vs End-to-End Learning Approaches} from Dario Mantegazza, developed at \gls{idsia} in Lugano. In his thesis, the author proposes a machine learning model for teaching a drone to interact with a person by continuously flying to face the user frontally, towards the direction of the head. The problem is approached as a reactive control procedure and addressed with supervised learning, thus provides an interesting starting point for many other robotics applications. 

The author collect data and test his model on the Parrot Bebop 2, a 500grams drone commonly used for photography and leisure purposes, capable of effective video stabilization. However, the software runs off-board, on a dedicated computer remotely connected through WiFi. 

A considerable amount of flights is recorded for building the training data by programmatically flying the drone in front of a person, controlling it through an omniscient controller which knows both the drone's and user's pose. Images produced by the front-facing camera of the drone are used as input for a custom designed \gls{resnet} architecture to infer the relative user's position \gls{wrt} the drone. Practically, the neural network performs a regression on the four variables that form the user's pose (X, Y, Z, YAW) and learns to predict their values by using spacial information contained into the input images. 

In the paper, the author also makes a comparison between the mediated approach described above and another end-to-end approach that directly learns control signals\footnote{desired pitch, roll, yaw and vertical velocity} for the drone, instead of the user's pose. Another experiments also considers a learned controller. All the solutions provide similar results, but the former can be adapted to other tasks by simply designing a custom controller, providing a more transparent and analyzable solution\footnote{take a look at image \ref{fig:proximitynet-architecture-paper3approaches} for details about models architectures designed for the three approaches}.

Even though this kind of problems on human recognition and pose estimation could be faced with more advanced deep learning algorithms, making a simple regression on four variables allows the network to be small, so that the prediction task is light, fast to execute, and possibly portable on low-end devices.

\medskip

Network and dataset defined in \cite{mantegazza2019visionbased} have been used for our entire project, so the original code repository\footnote{\url{https://github.com/idsia-robotics/proximity-quadrotor-learning}} is available for reference. Next chapters constantly make use of this particular model architecture, that will be further explained in section \ref{sec:proximitynet}. 

Having no official name, for enhancing readability, the custom \gls{resnet} architecture proposed by the author will be simply called \textit{ProximityNet}. For a better understanding, also a good descriptive video is available at \url{https://drive.switch.ch/index.php/s/MlEDrsuHcSl5Aw5}.



\subsection{Embedded Implementation of Controller for Nano-Drones}
\label{subsec:sota-nicky}

Autonomous navigation is an important and well-known area of research in robotics, which usually requires to accomplish complex and computationally-expensive tasks such as localization, mapping and path planning. Recent studies have started to approach autonomous driving through deep learning and imitation learning\cite{imitation-learning}, where neural networks learn by imitating human behavior in specific tasks. 

\medskip

In 2018, researchers at the UZH University of Zürich have demonstrated that \gls{resnet}s are able to provide satisfactory performance in the field (\cite{Loquercio_2018}). They developed DroNet, a forked \gls{cnn} that predicts, from a single gray-scale image, a steering angle and a collision probability. In other words, the model learns to steer and avoid obstacles from forward-looking videos recorded by cars and bikes while driving in real contexts. In this case, both the prediction and controller tasks were powered off-drone on a dedicated computer, remotely connected through WiFi. 

A year later, ETH Zürich was able to develop PULP-DroNet, porting the \gls{cnn} on the Crazyflie\footnote{\url{https://www.bitcraze.io/products/crazyflie-2-1/}}, a nano-drone with a size of only 3 $3 \times 3$ centimeters for a weight of 27 grams. They propose a general methodology for deploying on-board deep learning algorithms for  ultra-low-power devices \cite{palossi2019pulpdronetIoTJ}, without any needs of an external laptop to run the software.

\medskip

Inspired by PULP-DroNet, \gls{idsia} adapted its ProximityNet to work on-board the Crazyflie with excellent results \cite{zimmerman2020thesis}. The nano-drone is able to achieve good quantitative and qualitative performance, regardless any problem deriving from working with such low-end devices. Main challenges are represented by low computational power, energy consumption management, and low-fidelity camera with no video stabilization\footnote{Himax HM01B0 camera, able to produce $320 \times 320$ \gls{mp} at 60 FPS. However, frame rate is incredibly reduced during data collection due to platform limitation for image transfer.}.




\section{Network Interpretability}
\label{sec:network-interpretability}

\glsreset{ml}
\glsreset{nn}

Deep \gls{nn}s learn abstract representations for finding a logical mapping between their input and output, determined by well-defined mathematical computations which involve the input itself and the progressively learned network parameters. Inspired by biological brains, this approach seems to be incredibly effective on a huge variety of tasks.

Unlike other \gls{ml} techniques, \gls{nn} are known to produce "black-box" models, particularly hard to understand even from domain experts. Their reasoning and comprehension is intrinsic in the network parameters, which are nothing but numbers.

\medskip 

However, when working with real-world problems, it is extremely important to be able to explain what a \gls{ml} model is actually understanding. This is crucial for building trust in algorithms and to be sure there are no undesirable biases in the models, which could raise serious problems especially in critical fields such as medicine and law.

\gls{xai} is the field of study which tries to make \gls{ml} results, and their underlying basis for decision-making, properly understandable to humans (\cite{xai-wiki}). 

\medskip 

Regarding \gls{xai} for \gls{cnn}s, researchers has developed many techniques for understanding what a \gls{nn}s actually care of when producing an output based on an input image. Main efforts regard feature visualization and attribution, but recent advanced studies also shown how these methods can be used altogether (\cite{olah2018the}).

This section briefly explain these two major areas for \gls{cnn} interpretability, with a particular focus on spacial attribution, the chosen methodology for our work.



\subsection{Feature Visualization}

Sources: \cite{olah2017feature}



\subsection{Spatial Attribution with GradCAM}
\label{subsec:gradcam-theory}

Sources: \cite{Selvaraju_2019}, \cite{gradcam_medium}

\begin{figure}[!htb]
	\centering
	\includegraphics[width=0.9\textwidth]{"contents/images/gradcam/02-gradcam-schema"}
	\caption[\gls{gradcam} schematic functioning]{\gls{gradcam} schematic functioning (\cite{Selvaraju_2019})}
	\label{fig:gradcam-schema}
\end{figure}

\begin{figure}[!htb]
\centering
\includegraphics[width=0.9\textwidth]{"contents/images/gradcam/02-gradcam-catdog"}
\caption[\gls{gradcam} example on dog-cat classification]{\gls{gradcam} example on dog-cat classification (\cite{Selvaraju_2019})}
\label{fig:gradcam-catdog}
\end{figure}




\section{Network Generalization}
\label{sec:network-generalization}



\subsection{Data Augmentation}
\label{subsec:data-augmentation}



\subsection{Domain Randomization}
\label{subsec:domain-randomization}

\textbf{TODO}

Summary from \cite{mehta2019active}

See \cite{weng2019DR_explanation}
 
"Domain randomization is a popular technique for improving domain transfer, often used in a zero-shot setting when the target domain is unknown or cannot easily be used for training. In this work, we empirically examine the effects of domain randomization on agent generalization. Our experiments show that domain randomization may lead to suboptimal, high-variance policies, which we attribute to the uniform sampling of environment parameters."


\subsubsection{Virtual Simulation}
\label{subsec:virtual-simulation}

Imitation learning through simulation is recently becoming an interesting and successful approach for both reinforcement learning (\cite{imitation_learning_survey} and \cite{imitation_learning_3d_navigation}) object recognition (\cite{tobin2017domain}, \cite{weng2019DR}).

Robot and environment can be replicated through a dedicated simulator such as Gazebo\footnote{\url{http://gazebosim.org/}}, often used in robotics with \gls{ros}\footnote{\url{https://www.ros.org/}} due to its straightforward integration, or even with general-purpose graphic engines. Unreal Engine\footnote{\url{https://www.unrealengine.com/en-US/}} Unity\footnote{\url{https://unity.com/}} are well-known simulators designed for game-development, but recently used for \gls{vr} and \gls{ar} applications. They give developers unlimited possibilities, carefully supported by solid physics engines and active communities.

\medskip

Given the difficulty of collecting data for our task, exploring the possibility of simulating the entire scenario in a 3D virtual-world is intriguing, especially to replace the need of a complex \gls{mocap} system. Integrating odometry support, drone and people can be thoroughly modeled to act as in the real-world, with similar movements and sensing capabilities, in order to collect the data very efficiently. Virtual simulation gives both the opportunity of reproducing real indoor/outdoor scenes, but also randomizing the background with artificially generated textures.

Even though the approach appears to obtain sub-optimal results, a complete and adaptable implementation requires a lot of effort, yet unlocking a huge number of possibilities. Considering the amount of fine details to consider and issues that can arise during the development of such simulators, we opt instead to work with an easier generalization pipeline that mostly concerns machine learning only.


\subsubsection{Background Replacement with Mask R-CNN}
\label{subsec:sota-maskrcnn}

\textbf{TODO}

Previous works on images: \cite{yue2019domain}, \cite{Takahashi_2020}

MaskRCNN: \cite{he2018mask}, \cite{maskrcnn_explanation}, \cite{maskrcnn_arcgis}



