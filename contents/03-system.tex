\chapter{System Description}
\label{chap:system}

\glsreset{idsia}
\glsreset{wrt}
\glsreset{mocap}
\glsreset{mp}
\glsreset{ros}

We focus on improving the original work from \cite{mantegazza2019visionbased}, introduced in section \ref{subsec:sota-dario}. This chapter aims to provide a generic view of the system, starting from its main components and how they interact for flying and controlling the drone. Then, we present tools, libraries and dataset we are working with for achieving the goal of enhancing the model generalization capabilities.




\section{Hardware}
\label{sec:hardware}

Working from a machine learning perspective, our task does not require to interact with physical components. However, it is helpful to know the environment in which the drone was originally taught to fly. This is physically located in Manno (Switzerland) at \gls{idsia}\footnote{now relocated to Lugano}.



\subsection{Parrot Bebop Drone 2}
\label{subsec:bebop}

\begin{figure}[!htb]
	\centering
	\includegraphics[width=.8\textwidth]{"contents/images/03-Parrot-Bebop-2"}
	\caption[Parrot Bebop Drone 2]{Parrot Bebop Drone 2}
	\label{fig:bebop}
\end{figure}

The entire work is built around the Parrot Bebop Drone 2 (figure \ref{fig:bebop}), a lightweight drone (500 grams) with a size of $382 \times 328 \times 89$ millimeters. A 2700 mAh swappable battery gives power to four brushless motors and a dual-core processor with a quad-core GPU, for a maximum flight time of 25 minutes. Connectivity is provided through 2.4 GHz 802.11a/b/n/ac Wi-Fi that enables remote control via mobile app or Parrot Skycontroller (up to a distance of 2km).

The drone is equipped with many simultaneous sensors to compute velocities, orientation, altitude, and GPS coordinates, in order to ensure maximum stability during the whole flight. For this project, we mainly focus on its camera, able to shoot 14 \gls{mp} photos and record Full HD 1080p videos at 30 \gls{fps}. Even though the original \gls{fov} is 180°, raw camera images pass through a software stabilization that produces 16:9 images with a horizontal \gls{fov} of about 80°. The 3-axis digital stabilization technique implemented by Parrot is able to compensate for the drone's pitch and roll, in order to provide correct-oriented horizontal images and stable videos regardless of the drone's movements. Full specifications provided by the official \cite{bebop}.



\subsection{OptiTrack}
\label{subsec:optitrack}

To monitor the drone and the user's movement we require a \gls{mocap} system, that is able to record 3D coordinate of objects and people in space. The technique is widely used for motion tracking in a large variety of fields such as film making and animation, virtual reality, sport, medicine, and even military. A common way to implement a \gls{mocap} systems is by using special cameras placed around the area to be tracked. These are able to collect optical signals from passive\footnote{a passive marker reflect light} or active markers\footnote{an active marker emits its own light} inside the area.

\medskip

\gls{idsia} adopt \cite{optitrack}, which is producing real-time \gls{mocap} systems since 1996 and are today world’s choice for low-latency and high-precision 6 \gls{dof} tracking for ground and aerial robotics, both indoor and outdoor.



\subsection{Drone Arena}
\label{subsec:drone-arena}

At \gls{idsia}, a dedicated room has been equipped with a \gls{mocap} system composed by 12 OptiTrack Prime$^x$13 \gls{ir} cameras\footnote{1.3 \gls{mp}, 240 \gls{fps}, $\pm0.20$ mm 3D accuracy in a $9 \times 9$ meters area with 14mm markers} for medium-sized areas (figure \ref{fig:optitrack-camera}). They track the movements of passive markers, placed both on the person's head facing the drone and on the drone itself. Schematic and actual representation of the arena are shown in figures \ref{fig:optitrack-schema} and \ref{fig:drone-arena}. Such composition is able to track a theoretical number of 18 drones inside an available area of $6 \times 6$ meters (here surrounded by a safety net). A virtual fence of $4.8 \times 4.8$ meters virtually constraints the total area in which the drone is allowed to fly.

\begin{figure}[!htb]
	\centering
	\includegraphics[width=1\textwidth]{"contents/images/03-optitrack-schema"}
	\caption[Schematic OptiTrack system with 12 OptiTrack Prime cameras]{Schematic OptiTrack system with 12 OptiTrack Prime cameras}
	\label{fig:optitrack-schema}
\end{figure}

\begin{figure}[!htb]
	\begin{center}
		\begin{subfigure}[h]{0.29\textwidth}
			\centering
			\includegraphics[width=1\textwidth]{"contents/images/03-optitrack-camera"}
			\caption[]{Prime$^x$13 camera}
			\label{fig:optitrack-camera}
		\end{subfigure}
		\hfill
		\begin{subfigure}[h]{0.69\textwidth}
			\centering
			\includegraphics[width=1\textwidth]{"contents/images/03-arena"}
			\caption[]{Drone with passive markers (\cite{mantegazza2018thesis})}
			\label{fig:drone-arena}
		\end{subfigure}
	\end{center}
	\vspace{-0.5cm}
	\caption[Drone arena at \gls{idsia}]{Drone arena at \gls{idsia}}
\end{figure}




\section{Software}
\label{sec:software}

This section presents tools and software used to conduct our research for improving the existing system. The original work is written in Python 3 and based on ROS, TensorFlow 1, and Keras. We adapt the code for working with TensorFlow 2.

Our first experiments were carried out with Jupyter Notebooks via Google Colab on a GPU-accelerated runtime, while the final code is provided as Python scripts. For debugging, we use a Windows 10 laptop equipped with an NVIDIA GeForce GTX 950M graphic card, while training is performed on a dedicated Ubuntu 18.04 workstation available at \gls{idsia}, mounting four NVIDIA GeForce RTX 2080 Ti \footnote{anyway, multiple available GPUs are used for single-GPU computing, not simultaneously}.



\subsection{Frameworks}
\label{subsec:frameworks}

Here is a list of the main libraries which compose our software.

\paragraph*{\texttt{Numpy}}
Largely used in the whole project for computation on arrays. Numpy is the fundamental package for scientific computing in Python, that provides a multidimensional array object, various derived objects (such as masked arrays and matrices), and an assortment of routines for fast operations on arrays. 

\paragraph*{\texttt{Pickle}}
Mainly used for saving and loading Numpy arrays. The pickle module implements binary protocols for (de-)serializing Python object structures. 

\paragraph*{\texttt{Matplotlib}}
The first choice for building charts, visualize images or any kind of figure. Matplotlib is a comprehensive library for creating static, animated, and interactive visualizations in Python. Its \texttt{pyplot} module is inspired by MATLAB.

\paragraph*{\texttt{OpenCV}}
Mainly used for efficient image/video manipulation and visualization, together with Matplotlib. OpenCV is an open-source library that includes several hundreds of computer vision algorithms.

\paragraph*{\texttt{TensorFlow 2}}
The entire project strongly relies on \cite{tensorflow} (TF) from start to end: network interpretation, person masking, training, and quantitative evaluation. Created by the Google Brain team, TensorFlow is an open-source library for numerical computation and large-scale machine learning. It can be used across a range of tasks but has a particular focus on training and inference of deep neural networks. In version 2, it introduces a lot of comforts for easier development with a less steep learning curve.

\paragraph*{\texttt{Keras}}
Used for defining the network architecture, training, and evaluating the model. \cite{keras} is the high-level API of TensorFlow 2: an approachable, highly-productive interface for solving machine learning problems, with a focus on modern deep learning. It provides essential abstractions and building blocks for developing and shipping machine learning solutions with high iteration velocity.

\paragraph*{\texttt{TensorBoard}}
Used with TensorFlow to precisely profile data generator performance for optimizing training time on the GPU. \cite{tensorBoard} is a tool for providing the measurements and visualizations needed during the machine learning workflow. It enables tracking experiment metrics, visualizing the model graph, and much more.

\paragraph*{\texttt{Sklearn}}
Only used for automatically computing some evaluation metrics. Sklearn is a simple and efficient tool for predictive data analysis reusable in various contexts built on NumPy, SciPy, and Matplotlib.

\paragraph*{\texttt{Albumentation}}
Used during training for implementing classic image augmentation. \cite{albumentations} efficiently implements a rich variety of image transform operations that are optimized for performance through a concise yet powerful interface. Widely used in industry, deep learning research, machine learning competitions, and open source projects.

\paragraph*{\texttt{tf-keras-vis}}
Used for applying GradCAM and other interpretability techniques. Open-source library for network interpretation, available on GitHub thanks to \cite{tf-keras-vis}. Derived from the original keras-vis (\cite{keras-vis}) high-level toolkit for visualizing and debugging trained Keras neural network models.

\paragraph*{\texttt{akTwelve Mask\_RCNN}}
Used for human detection and segmentation in background replacement. Open-source implementation of Mask R-CNN on Python 3, Keras, and TensorFlow 2 available on GitHub thanks to \cite{MaskRCNN_akTwelve}. The model generates bounding boxes, segmentation masks, and categorization labels for each instance of an object in the image.

\paragraph*{\texttt{\gls{ros}}}
Even tough not used in our work, it is crucial in \cite{mantegazza2019visionbased} to actually control the drone during flight. \cite{ROS} is an open-source robotics middleware suite\footnote{available for C++, Python, and Lisp} for building robot applications, providing hardware abstraction, implementation of commonly-used functionality, message-passing between processes, and package management. It also integrates with additional tools, such as \cite{rviz} and \cite{Gazebo}, for real-time 3D visualizations and simulations.




\section{Data}
\label{sec:dataset}

This section refers to the dataset acquired for \cite{mantegazza2019visionbased}, which is also used for our research.


\subsection{Collection}
\label{subsec:control}

Data have been entirely collected in the dedicated drone arena, presented in section \ref{subsec:drone-arena}. A good dataset should ideally provide images from various scenarios, but such a kind of data is not easy to record. The ground truth must be acquired by a complex and expensive \gls{mocap} system, particularly difficult to be moved and reassembled outdoor.

\medskip

The drone is controlled by a \gls{ros} script, running on a dedicated computer remotely connected through WiFi. It relies on the known user's pose - from now on, the \textit{target pose} (i.e., the pose of the user seen by the drone reference frame) - to compute acceleration commands for the drone. The script is responsible of making the drone hovering in front of the person, facing the head orientation at a predefined 1.5 meters distance.

During data collection, both user's and drone's poses are captured by the OptiTrack using proper markers, placed on the drone and on the person's head (picture \ref{fig:drone-facing}). The target poses over time\footnote{mathematically computed by a script from \cite{mantegazza2018thesis}}, are synchronized with the video stream coming from the camera, and saved into \texttt{rosbag} files. Figure \ref{fig:drone-demo-2} shows an illustration of the system from a bird-eye view.

\begin{figure}[!h]
	\centering
	\includegraphics[width=0.5\textwidth]{"contents/images/03-drone-facing"}
	\caption[Markers placed on top of drone and user's head]{Markers placed on top of drone and user's head (\cite{mantegazza2018thesis})}
	\label{fig:drone-facing}
\end{figure}

\begin{figure}[!htb]
	\centering
	\includegraphics[width=1\textwidth]{"contents/images/03-arena-demo-2"}
	\caption[OptiTrack and data collection illustration]{OptiTrack and data collection illustration (\cite{mantegazza2018thesis})}
	\label{fig:drone-demo-2}
\end{figure}



\subsection{Composition}

Data collected inside the arena have been used to build the dataset for training a machine learning model, able to infer the target pose from a picture. For building both the training and the testing set, several flight sessions have been recorded using the omniscient controller described above. 

The dataset contains a total of 13 different people, which differs both in physical characteristics and outfit, moving in different ways under various (artificial) light conditions. Many objects are present in the background of recorded images, and some experiments involve more than one person in front of the drone\footnote{anyway, the drone always had to follow the nearest user (equipped with OptiTrack markers)}. In total, 45 minutes of usable videos were used to compose the dataset, which counts about 63'000 and 11'000 frames respectively for training and testing sets.

A complete overview of images composing the training set is shown in figure \ref{fig:frontalnet-dataset-overview}. Please notice that a few frames in the dataset are affected by digital artifacts, mainly caused by connection issues during video recording (figure \ref{fig:frontalnet-dataset-glitch}); also, there are frames in which no person is present at all, because of particular movements sequences during which the drone actually lose the user (figure \ref{fig:frontalnet-dataset-missing}).

\begin{figure}[!htb]
	\centering
	\includegraphics[width=0.4\textwidth]{"contents/images/03-data-glitch"}
	\caption[A frame with digital artifact caused by connection issues]{A frame with digital artifact caused by connection issues}
	\label{fig:frontalnet-dataset-glitch}
\end{figure}

\clearpage
\vspace*{5ex}
\begin{figure}[!h]
	\centering
	\includegraphics[width=1\textwidth]{"contents/images/03-data-overview"}
	\caption[A complete overview of images in the training set]{A complete overview of images in the training set}
	\label{fig:frontalnet-dataset-overview}
\end{figure}
\clearpage

\begin{figure}[!htb]
\centering
\includegraphics[width=1\textwidth]{"contents/images/03-data-missing"}
\caption[A movements sequence which led to images with no person presents]{A movements sequence which led to images with no person presents}
\label{fig:frontalnet-dataset-missing}
\end{figure}

The ground truth is represented by the four variables, associated with each captured image. The variables explain the user's pose \gls{wrt} the drone, and their interpretation is described here:

\begin{itemize}
	\item \texttt{X} is the distance of the user from the drone and affects the pitch (acceleration along the X-axis). Usually, the drone flies at 1.5 meters from the user.
	\item \texttt{Y} represents the horizontal alignment of the user in front of the drone and affects the roll (acceleration along the Y-axis). When the user is horizontally centered in front of the drone, this variable will be equal to 0.
	\item \texttt{Z} represents the vertical alignment of the user in front of the drone and affects the velocity along the Z-axis. When the user is vertically centered in front of the drone, this variable will be equal to 0.
	\item \texttt{W} represents the angle created between head's pointing direction and drone position, is influenced by head orientation, and affects the yaw (angular velocity around the Z-axis). If the user is perfectly facing the drone, this variable will be equal to 0.
\end{itemize}

From the distribution of the variables in the training set, shown in figure \ref{fig:frontalnet-dataset-distribution-regr}, we notice that most of the time the user is somehow centered in the image. This is an effect caused by the \gls{ros} controller based on known poses. The variation of the variables is affected by the user's movements in space, the more sudden they are, the greater the deviation.

\begin{figure}[!h]
	\begin{center}
		\begin{subfigure}[h]{0.24\textwidth}
			\centering
			\includegraphics[width=1\textwidth]{"contents/images/distributions/x-regr"}
		\end{subfigure}
		\hfill
		\begin{subfigure}[h]{0.24\textwidth}
			\centering
			\includegraphics[width=1\textwidth]{"contents/images/distributions/y-regr"}
		\end{subfigure}
		\hfill
		\begin{subfigure}[h]{0.24\textwidth}
			\centering
			\includegraphics[width=1\textwidth]{"contents/images/distributions/z-regr"}
		\end{subfigure}
		\hfill
		\begin{subfigure}[h]{0.24\textwidth}
			\centering
			\includegraphics[width=1\textwidth]{"contents/images/distributions/w-regr"}
		\end{subfigure}
	\end{center}
	\vspace{-0.5cm}
	\caption[Target variables distribution for the regression task]{Target variables distribution for the regression task}
	\label{fig:frontalnet-dataset-distribution-regr}
\end{figure}


