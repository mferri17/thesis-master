\chapter{Solution Design}
\label{chap:design}

\glsreset{ai}
\glsreset{nn}
\glsreset{cnn}

This chapter explores the existing model's issues, proposes a solution, and presents initial experiments on its feasibility.




\section{Problem Summary}
\label{sec:frontalnet-generalization}

In section \ref{subsec:sota-dario}, we introduce the original paper we are working on and present its architecture and basic performance. Chapter \ref{chap:system} illustrates environment composition and presents the dataset used to train the machine learning model, as declared in Mantegazza et al. \cite{mantegazza2019visionbased}.

\medskip

As explained in section \ref{subsec:frontalnet-performance}, FrontalNet achieved quite good performance on the test set. Nevertheless, its behavior must be proven on the real drone to certify the model's usability. \cite{mantegazza2019visionbased} reports experiments conducted inside the arena by flying the drone without the \gls{mocap} system, only relying on the learned model for computing the user's pose. The outcome is excellent, with the drone actually performing its task without any issues\footnote{see figure \ref{fig:frontalnet-trajectories} in the appendix for further details}.

However, both quantitative and qualitative evaluations have been carried out using a set of images similar to the training one. Even though the model predictions were good with unknown users, they still move in the same drone arena where the training data have been collected. For a complete analysis, we must consider model performance in unknown environments.

The official paper does not talk about the topic, but during a direct discussion with the author, we discovered that flying performance outside of the drone arena was not consistent with usual model behavior. The drone was not able to follow the user appropriately, and its movements were unexplainable. We conclude that the \gls{cnn} is not able to generalize the task when outside of the environment it already knows.

\medskip

Our goal is to explore ways of improvement to generalize the model, allowing it to theoretically predict the user's pose in any other unknown scenario. The next sections first try to understand the neural network's main issues and limitations and then provide a solution for the generalization problem.




\section{Model Interpretation with Grad-CAM}
\label{sec:model-interpretration}

\glsreset{nn}
\glsreset{gradcam}

In the previous section, we discussed insufficient experimental results obtained by FrontalNet in predicting the user's pose in an unknown environment (i.e., outside of the drone arena). This section highlights the main issues behind the lack of generalization capabilities by understanding what the model is actually learning.

\medskip

Convolutional Neural Networks (CNN) are suited for computer vision because of their ability to extract spatial-related insights from images. As any other \gls{nn}, even \gls{cnn}s are "black-boxes". This means that their internal behavior is particularly challenging for humans to understand.

Among network interpretability techniques introduced in section \ref{sec:network-interpretability}, we choose \gls{gradcam}. The algorithm is indeed the most understandable way of visualizing what a \gls{cnn} is actually seeing.

As explained in section \ref{subsec:gradcam-theory}, \gls{gradcam} is able to effectively visualize parts of an input image which are actually responsible for predicting a certain output\footnote{for an easy understandable \gls{gradcam} example, please refer to the figure \ref{fig:gradcam-catdog} which regards a simple dogs VS cats classifier}.

\medskip

Research in the field is still on-going, and most of the available resources are for TensorFlow 1. The most powerful and famous library for network interpretability is \texttt{Lucid} \cite{tf-lucid}, from the official TensorFlow team. Since Lucid does not provide native support with Keras models, we prefer to use instead the \texttt{tf-keras-vis} library \cite{tf-keras-vis} for TensorFlow 2.



\subsection{Regression to Classification}
\label{subsec:gradcam-regrtoclass}

\gls{gradcam} is designed to be applied on classification tasks rather than regression ones. Even though a porting of the algorithm for regression has been published (Regression Activation Map \cite{wang2019diabetic}), it appears to be an isolated case. For this reason, and for network interpretation only, we decide to transform our problem into a classification task.

The ground truth is composed of four variables with specific domains. Figure \ref{fig:frontalnet-dataset-distribution-regr} in the previous chapter shows their distribution. Every variable has a particular "central" value obtained when the user is centered in the image. 

We decide to split continuous values into three different classes, which account for values smaller, around, and higher than the "center". We call these buckets respectively \texttt{low}, \texttt{medium}, and \texttt{high}.

\begin{itemize}
	\item \texttt{X} values are splitted at $1.4$ and $1.6$
	\item \texttt{Y} values are splitted at $-0.15$ and $+0.15$
	\item \texttt{Z} values are splitted at $-0.05$ and $+0.05$
	\item \texttt{W} values are splitted at $-0.20$ and $+0.20$
\end{itemize}

So, for example, \texttt{X} values greater than $1.6$ will be classified as \texttt{high}, while \texttt{Z} values between $-0.05$ and $+0.05$ will be classified as \texttt{medium}. These intervals have been defined to have a uniform class distribution over the training set (figure \ref{fig:frontalnet-dataset-distribution-class}).

\begin{figure}[!h]
	\begin{center}
		\begin{subfigure}[h]{0.24\textwidth}
			\centering
			\includegraphics[width=1\textwidth]{"contents/images/distributions/x-class"}
		\end{subfigure}
		\hfill
		\begin{subfigure}[h]{0.24\textwidth}
			\centering
			\includegraphics[width=1\textwidth]{"contents/images/distributions/y-class"}
		\end{subfigure}
		\hfill
		\begin{subfigure}[h]{0.24\textwidth}
			\centering
			\includegraphics[width=1\textwidth]{"contents/images/distributions/z-class"}
		\end{subfigure}
		\hfill
		\begin{subfigure}[h]{0.24\textwidth}
			\centering
			\includegraphics[width=1\textwidth]{"contents/images/distributions/w-class"}
		\end{subfigure}
	\end{center}
	\vspace{-0.5cm}
	\caption[Target variables distribution for the classification task]{Target variables distribution for the classification task}
	\label{fig:frontalnet-dataset-distribution-class}
\end{figure}



\subsection{Re-training}
\label{subsec:gradcam-retrain}

Considering the new ground truth, we define a new model architecture by replacing regression outputs with classification ones. We completely re-train the new model, by using the \texttt{categorical\_crossentropy} loss and the \texttt{accuracy} metric, both suited for multi-class problems. As the original FrontalNet model, we opt for an \gls{adam} optimizer and a base learning rate of $0.001$, progressively reduced on validation loss plateaus. 

Results are shown in figure \ref{fig:gradcam-retrain-simple}. After 30 epochs, the charts report a loss slightly smaller than 1, both for training and validation, and accuracy over the 80\% for all the variables. These values are not ideal for drawing a conclusion but have instead been used for comparing different training experiments conducted on the new classification model.

\begin{figure}[!h]
	\centering
	\includegraphics[width=1\textwidth]{"contents/images/04-metrics-class-simple"}
	\caption[\gls{gradcam}: loss and accuracy of the new classification model]{\gls{gradcam}: loss and accuracy of the new classification model}
	\label{fig:gradcam-retrain-simple}
\end{figure}



\subsection{Interpretation Issues}
\label{subsec:gradcam-reading}

A proper understanding of \gls{gradcam} results requires a thorough ability on reading the following charts\footnote{Basic knowledge on how the related library works would also be helpful. Tutorial: \url{https://github.com/keisen/tf-keras-vis/blob/8f83773520069367902becc0a668dda90ab76349/examples/attentions.ipynb}}.
 
\medskip

To run \gls{gradcam}, it is required to specify a class for which to compute the respective activation mapping. In our case, the class would be one of the three defined in section \ref{subsec:gradcam-regrtoclass}: \texttt{low}, \texttt{medium} or \texttt{high}. 

When working on standard classification problems, things are pretty easy. If we classify animals, we indicate \gls{gradcam} a specific animal class (e.g., \texttt{lion}), and the algorithm will provide a heatmap that overlays the portion of the image which is mainly associated with that animal (e.g., hopefully, the lion will be highlighted, if it is present in the image).

However, our \gls{ml} does not look like a standard classification problem since it has been adapted from a multi-output regression task. Classes only serve as categorical values for variables that are actually numerical, and this can introduce some issues when analyzing \gls{gradcam} visualizations.

\renewcommand{\labelenumi}{\Alph{enumi}}
\begin{enumerate}
    \item For a certain input image, there is no particular portion of the image that can be strictly associated with one or another class in particular. Therefore, assuming to have a correctly trained model and considering a single image, we do not expect the heatmap generated by \gls{gradcam} for a certain class (e.g., \texttt{low}) to be different from heatmaps generated for the other classes (e.g., \texttt{medium} and \texttt{high}). The discriminator for the model's predictions must always be the user; thus, also \gls{gradcam} should only focus on the user.
    \item As the regression task, also the classification one predicts four variables that correspond to the user's coordinates. A multi-output network further introduces another complexity on reading \gls{gradcam} results since the algorithm can be run differently on each variable, producing different heatmaps. Once again, if the model is correct, we expect that \gls{gradcam} will produce the same output regardless of the inspected variable (i.e., it always highlights the user in the images).
    \item In some cases, the network predictions for one or more variables are different from the actual value (i.e., the ground truth). When examining \gls{gradcam} results corresponding to erroneously estimated values, both the predicted and the actual class have to be considered to understand better what is going wrong with the model.
\end{enumerate}

Analyzing \gls{gradcam} results, we want to confirm the hypothesis made in A and B. If heatmaps mainly overlay people in the images, we can conclude that the \gls{cnn} can effectively understand the concept of a person and produce its outputs based on the user's position only. Otherwise, if \gls{gradcam} reveals that other parts of the images are considered important by the model, then we realize that the predictions are driven by undesired factors (e.g., objects in the background).

\medskip

Figure \ref{fig:gradcam-example-1} displays a typical example on \gls{gradcam} application with a thorough division in variables (\texttt{X, Y, Z, W}) and classes (\texttt{low, medium, high}).

\begin{figure}[!h]
	\centering
	\includegraphics[width=1\textwidth]{"contents/images/gradcam/04-gradcam-example-1"}
	\caption[\gls{gradcam}: example of application for each variable and class]{\gls{gradcam}: example of application for each variable and class}
	\label{fig:gradcam-example-1}
\end{figure}

Heatmaps are not easy to interpret due to a large variety of parameters to consider. As a guideline, it is possible to only consider, for each variable, the column which corresponds to the predicted class (or the actual, if it differs). Actual and predicted values are available in the right-most parenthesis, as "GT" and "PR" respectively. Rows define variables, while columns stand for the classes. It is clearly visible how no specific correlation in \gls{gradcam} results is available between variables, classes, and computed predictions.

By calling \gls{gradcam} without specifying a particular class, it is also possible to obtain an average result on all classes, shown in the last column. Applying similar reasoning also on variables, we can produce a single meaningful image. This represents a \gls{gradcam} global average, which is computed on every variable and class at once (figure \ref{fig:gradcam-example-2}).

\begin{figure}[!h]
	\centering
	\includegraphics[width=0.5\textwidth]{"contents/images/gradcam/04-gradcam-example-2"}
	\caption[\gls{gradcam}: example of application on a global average]{\gls{gradcam}: example of application on a global average}
	\label{fig:gradcam-example-2}
\end{figure}



\subsection{Results}
\label{subsec:gradcam-results}

As already shown in figure \ref{fig:gradcam-example-1}, it seems that the network is not only considering the person in the frame for computing its output but instead relies on the whole image with particular attention on some spots.

From the previous section, we understand that reasoning with \gls{gradcam} heatmaps is not trivial, and separating visualization by variables and classes is not totally convenient when we can plot the global average \gls{gradcam} instead. For simplicity, this section will only focus on single-image results. However, full \gls{gradcam} visualizations are available in the appendix \ref{sec:extra-gradcam} for further inspection.

\subsubsection*{Reasonable detections}

Figure \ref{fig:gradcam-ok} displays examples of correctly working scenarios in which the person is well-detected by \gls{gradcam}. In these situations, it often happens that the entire user is highlighted, but sometimes only the body or the head gets proper attention.

\begin{figure}[!h]
	\begin{center}
		\begin{subfigure}[h]{0.24\textwidth}
			\centering
			\includegraphics[width=1\textwidth]{"contents/images/gradcam/gradcam-ok-1"}
		\end{subfigure}
		\hfill
		\begin{subfigure}[h]{0.24\textwidth}
			\centering
			\includegraphics[width=1\textwidth]{"contents/images/gradcam/gradcam-ok-2"}
		\end{subfigure}
		\hfill
		\begin{subfigure}[h]{0.24\textwidth}
			\centering
			\includegraphics[width=1\textwidth]{"contents/images/gradcam/gradcam-ok-3"}
		\end{subfigure}
		\hfill
		\begin{subfigure}[h]{0.24\textwidth}
			\centering
			\includegraphics[width=1\textwidth]{"contents/images/gradcam/gradcam-ok-4"}
		\end{subfigure}
	\end{center}
	\vspace{-0.5cm}
	\caption[\gls{gradcam}: Correctly detected people]{\gls{gradcam}: Correctly detected people}
	\label{fig:gradcam-ok}
\end{figure}

Such precise results are not the standard. In many cases, the network focus is unstable, and the heatmaps frequently go in and out of the target person. The two sequences of frames shown in figures \ref{fig:gradcam-seq-dario} and \ref{fig:gradcam-seq-beard} fairly describe the usual behavior of the model seen by \gls{gradcam}.

\begin{figure}[!h]
	\begin{center}
		\begin{subfigure}[h]{0.24\textwidth}
			\centering
			\includegraphics[width=1\textwidth]{"contents/images/gradcam/gradcam-dario-1"}
		\end{subfigure}
		\hfill
		\begin{subfigure}[h]{0.24\textwidth}
			\centering
			\includegraphics[width=1\textwidth]{"contents/images/gradcam/gradcam-dario-2"}
		\end{subfigure}
		\hfill
		\begin{subfigure}[h]{0.24\textwidth}
			\centering
			\includegraphics[width=1\textwidth]{"contents/images/gradcam/gradcam-dario-3"}
		\end{subfigure}
		\hfill
		\begin{subfigure}[h]{0.24\textwidth}
			\centering
			\includegraphics[width=1\textwidth]{"contents/images/gradcam/gradcam-dario-4"}
		\end{subfigure}
	\end{center}
	\vspace{-0.5cm}
	\caption[\gls{gradcam}: Sequence transitioning from wrong to correct detections]{\gls{gradcam}: Sequence transitioning from wrong to correct detections}
	\label{fig:gradcam-seq-dario}
\end{figure}

\begin{figure}[!h]
	\begin{center}
		\begin{subfigure}[h]{0.24\textwidth}
			\centering
			\includegraphics[width=1\textwidth]{"contents/images/gradcam/gradcam-beard-01"}
		\end{subfigure}
		\hfill
		\begin{subfigure}[h]{0.24\textwidth}
			\centering
			\includegraphics[width=1\textwidth]{"contents/images/gradcam/gradcam-beard-02"}
		\end{subfigure}
		\hfill
		\begin{subfigure}[h]{0.24\textwidth}
			\centering
			\includegraphics[width=1\textwidth]{"contents/images/gradcam/gradcam-beard-03"}
		\end{subfigure}
		\hfill
		\begin{subfigure}[h]{0.24\textwidth}
			\centering
			\includegraphics[width=1\textwidth]{"contents/images/gradcam/gradcam-beard-04"}
		\end{subfigure}
		\vfill
		\begin{subfigure}[h]{0.24\textwidth}
			\centering
			\includegraphics[width=1\textwidth]{"contents/images/gradcam/gradcam-beard-05"}
		\end{subfigure}
		\hfill
		\begin{subfigure}[h]{0.24\textwidth}
			\centering
			\includegraphics[width=1\textwidth]{"contents/images/gradcam/gradcam-beard-06"}
		\end{subfigure}
		\hfill
		\begin{subfigure}[h]{0.24\textwidth}
			\centering
			\includegraphics[width=1\textwidth]{"contents/images/gradcam/gradcam-beard-07"}
		\end{subfigure}
		\hfill
		\begin{subfigure}[h]{0.24\textwidth}
			\centering
			\includegraphics[width=1\textwidth]{"contents/images/gradcam/gradcam-beard-08"}
		\end{subfigure}
		\vfill
		\begin{subfigure}[h]{0.24\textwidth}
			\centering
			\includegraphics[width=1\textwidth]{"contents/images/gradcam/gradcam-beard-09"}
		\end{subfigure}
		\hfill
		\begin{subfigure}[h]{0.24\textwidth}
			\centering
			\includegraphics[width=1\textwidth]{"contents/images/gradcam/gradcam-beard-10"}
		\end{subfigure}
		\hfill
		\begin{subfigure}[h]{0.24\textwidth}
			\centering
			\includegraphics[width=1\textwidth]{"contents/images/gradcam/gradcam-beard-11"}
		\end{subfigure}
		\hfill
		\begin{subfigure}[h]{0.24\textwidth}
			\centering
			\includegraphics[width=1\textwidth]{"contents/images/gradcam/gradcam-beard-12"}
		\end{subfigure}
	\end{center}
	\vspace{-0.5cm}
	\caption[\gls{gradcam}: Sequence of unstable detections going in and out of the person]{\gls{gradcam}: Sequence of unstable detections going in and out of the person}
	\label{fig:gradcam-seq-beard}
\end{figure}

\subsubsection*{Problematic detections}

The examples presented above mostly reflect the model expected behavior. However, our network interpretation also reveals many flaws in the prediction task. \gls{gradcam} exhibits several situations in which the model output is affected by recurrent elements in the dataset.

\begin{itemize}
	\item Objects in the background are prone to be considered important (figure \ref{fig:gradcam-background})
	\item Curtains seem often particularly attractive (figure \ref{fig:gradcam-curtains})
	\item Many parts of the room can easily distract the model, such as borders and baseboards or even blank spots on the walls (figure \ref{fig:gradcam-random})
	\item When dealing with multiple people in front of the camera, sometimes not only the nearest person is considered (figure \ref{fig:gradcam-double})
	\item Artificial glitches, caused by connection issues, are sometimes correctly ignored, and other times a source of distraction (figure \ref{fig:gradcam-glitch})
\end{itemize}

\begin{figure}[!h]
	\begin{center}
		\begin{subfigure}[h]{0.24\textwidth}
			\centering
			\includegraphics[width=1\textwidth]{"contents/images/gradcam/gradcam-background-1"}
		\end{subfigure}
		\hfill
		\begin{subfigure}[h]{0.24\textwidth}
			\centering
			\includegraphics[width=1\textwidth]{"contents/images/gradcam/gradcam-background-2"}
		\end{subfigure}
		\hfill
		\begin{subfigure}[h]{0.24\textwidth}
			\centering
			\includegraphics[width=1\textwidth]{"contents/images/gradcam/gradcam-background-3"}
		\end{subfigure}
		\hfill
		\begin{subfigure}[h]{0.24\textwidth}
			\centering
			\includegraphics[width=1\textwidth]{"contents/images/gradcam/gradcam-background-4"}
		\end{subfigure}
	\end{center}
	\vspace{-0.5cm}
	\caption[\gls{gradcam}: Objects in the background detected]{\gls{gradcam}: Objects in the background detected}
	\label{fig:gradcam-background}
\end{figure}

\begin{figure}[!h]
	\begin{center}
		\begin{subfigure}[h]{0.24\textwidth}
			\centering
			\includegraphics[width=1\textwidth]{"contents/images/gradcam/gradcam-curtains-1"}
		\end{subfigure}
		\hfill
		\begin{subfigure}[h]{0.24\textwidth}
			\centering
			\includegraphics[width=1\textwidth]{"contents/images/gradcam/gradcam-curtains-2"}
		\end{subfigure}
		\hfill
		\begin{subfigure}[h]{0.24\textwidth}
			\centering
			\includegraphics[width=1\textwidth]{"contents/images/gradcam/gradcam-curtains-3"}
		\end{subfigure}
		\hfill
		\begin{subfigure}[h]{0.24\textwidth}
			\centering
			\includegraphics[width=1\textwidth]{"contents/images/gradcam/gradcam-curtains-4"}
		\end{subfigure}
	\end{center}
	\vspace{-0.5cm}
	\caption[\gls{gradcam}: Curtains often distract the model]{\gls{gradcam}: Curtains often distract the model}
	\label{fig:gradcam-curtains}
\end{figure}

\begin{figure}[!h]
	\begin{center}
		\begin{subfigure}[h]{0.24\textwidth}
			\centering
			\includegraphics[width=1\textwidth]{"contents/images/gradcam/gradcam-random-1"}
		\end{subfigure}
		\hfill
		\begin{subfigure}[h]{0.24\textwidth}
			\centering
			\includegraphics[width=1\textwidth]{"contents/images/gradcam/gradcam-random-2"}
		\end{subfigure}
		\hfill
		\begin{subfigure}[h]{0.24\textwidth}
			\centering
			\includegraphics[width=1\textwidth]{"contents/images/gradcam/gradcam-random-3"}
		\end{subfigure}
		\hfill
		\begin{subfigure}[h]{0.24\textwidth}
			\centering
			\includegraphics[width=1\textwidth]{"contents/images/gradcam/gradcam-random-4"}
		\end{subfigure}
	\end{center}
	\vspace{-0.5cm}
	\caption[\gls{gradcam}: Model get easily distracted by various elements]{\gls{gradcam}: Model get easily distracted by various elements}
	\label{fig:gradcam-random}
\end{figure}

\begin{figure}[!h]
	\begin{center}
		\begin{subfigure}[h]{0.24\textwidth}
			\centering
			\includegraphics[width=1\textwidth]{"contents/images/gradcam/gradcam-double-1"}
		\end{subfigure}
		\hfill
		\begin{subfigure}[h]{0.24\textwidth}
			\centering
			\includegraphics[width=1\textwidth]{"contents/images/gradcam/gradcam-double-2"}
		\end{subfigure}
		\hfill
		\begin{subfigure}[h]{0.24\textwidth}
			\centering
			\includegraphics[width=1\textwidth]{"contents/images/gradcam/gradcam-double-3"}
		\end{subfigure}
		\hfill
		\begin{subfigure}[h]{0.24\textwidth}
			\centering
			\includegraphics[width=1\textwidth]{"contents/images/gradcam/gradcam-double-4"}
		\end{subfigure}
	\end{center}
	\vspace{-0.5cm}
	\caption[\gls{gradcam}: Detections when two people are present in the image]{\gls{gradcam}: Detections when two people are present in the image}
	\label{fig:gradcam-double}
\end{figure}

\begin{figure}[!h]
	\begin{center}
		\begin{subfigure}[h]{0.24\textwidth}
			\centering
			\includegraphics[width=1\textwidth]{"contents/images/gradcam/gradcam-glitch-1"}
		\end{subfigure}
		\hfill
		\begin{subfigure}[h]{0.24\textwidth}
			\centering
			\includegraphics[width=1\textwidth]{"contents/images/gradcam/gradcam-glitch-2"}
		\end{subfigure}
		\hfill
		\begin{subfigure}[h]{0.24\textwidth}
			\centering
			\includegraphics[width=1\textwidth]{"contents/images/gradcam/gradcam-glitch-3"}
		\end{subfigure}
		\hfill
		\begin{subfigure}[h]{0.24\textwidth}
			\centering
			\includegraphics[width=1\textwidth]{"contents/images/gradcam/gradcam-glitch-4"}
		\end{subfigure}
	\end{center}
	\vspace{-0.5cm}
	\caption[\gls{gradcam}: Model reactions to artificial glitches]{\gls{gradcam}: Model reactions to artificial glitches}
	\label{fig:gradcam-glitch}
\end{figure}



\subsection{Summary}
\label{subsec:gradcam-summary}

\gls{gradcam} results demonstrate that the model is not robust enough to only focus on the user who is actually facing the drone's camera. Instead, various portions of the input images appear to be taken into consideration when the model makes its predictions. Many distractors are coming from the background.

In light of this, we can reasonably assume that the \gls{resnet} has undesirably learned some details about the drone arena in which the dataset has been collected. The issue is common with supervised learning, and it is called overfitting. It happens when a \gls{ml} model learns much information that only belongs to the training set, being not able to properly work later on previously unseen data.

This is most likely the reason why the model is unable to control the drone outside of the arena, as previously discussed in section \ref{sec:frontalnet-generalization}.




\section{Person Masking}
\label{sec:masking}

From \gls{gradcam} results presented in the previous section, we conclude that the model is not capable of generalization. We have demonstrated that the main cause of the problem is inherent in the drone arena; thus, we would like to remove this factor (i.e., the room itself) from the equation.

We propose a solution, inspired by domain randomization, that consists of performing advanced data augmentation on the training data. Considering the images that compose our dataset, we only keep the user facing the drone and remove the background by randomly replacing it with something else. This section explores various algorithms for creating the mask of a person in an image. After several experiments, the solution we have decided to adopt is Mask R-CNN (section \ref{subsec:masking-maskrcnn}).



\subsection{Canny}
\label{subsec:masking-canny}

The first experiments are based on a classic computer vision technique called Canny Edge Detection \cite{canny1986}. A custom algorithm\footnote{adapted from \url{https://stackoverflow.com/a/29314286/10866825}} applies the related function from OpenCV \cite{opencv_canny} to find the edges inside the image, which are then used for also finding the contours. Only the biggest contour is taken into consideration for building a mask around the subject in the image.

\medskip

A core aspect of the Canny function, in order to find appropriate contours, is the choice of its parameters \texttt{minVal} and \texttt{maxVal}. These are used by the algorithm for distinguishing between \textit{sure-edges}, \textit{probable-edges} and \textit{no-edges}. Several experiments have been done with different values, but no combination of the two parameters is optimal on our dataset.

Figure \ref{fig:canny-overview} shows what happens with \texttt{minVal} and \texttt{maxVal} respectively set to 100 and 400. Most of the time, the person is well-detected, while other times, it completely disappears or even results in a fatal error (red frames). The room baseboard (the line between the floor and the wall) is often still present in the image, while many samples seem to preserve a huge portion of the background.

\begin{figure}[!htb]
	\centering
	\includegraphics[width=1\textwidth]{"contents/images/04-1canny-overview"}
	\caption[Canny edge detection overview on the training set]{Canny edge detection overview on the training set}
	\label{fig:canny-overview}
\end{figure}

In some cases, it even happens that the person's body is present in the image while the face disappears. For mitigating this problem, an enhanced version of the algorithm has been considered. This version is designed to always keep the person's face and part of the body in the resulting image, assuming their positions are known. Figure \ref{fig:canny-enanhced} illustrates the problem and demonstrates that results obtained from the enhanced version are still not acceptable since we cannot appropriately remove the scene's background.

\begin{figure}[!h]
	\begin{center}
		\begin{subfigure}[h]{1\textwidth}
			\centering
			\includegraphics[width=1\textwidth]{"contents/images/04-1canny-enhance-1"}
        % 	\caption[]{The face sometimes disappear}
		\end{subfigure}
		\vfill
		\begin{subfigure}[h]{1\textwidth}
			\centering
			\includegraphics[width=1\textwidth]{"contents/images/04-1canny-enhance-2"}
        % 	\caption[]{We design an area which must be kept}
		\end{subfigure}
		\vfill
		\begin{subfigure}[h]{1\textwidth}
			\centering
			\includegraphics[width=1\textwidth]{"contents/images/04-1canny-enhance-3"}
        % 	\caption[]{Final results}
		\end{subfigure}
	\end{center}
	\vspace{-0.5cm}
	\caption[Canny edge enhanced algorithm demonstration]{Canny edge enhanced algorithm demonstration}
	\label{fig:canny-enanhced}
\end{figure}



\subsection{GrabCut}
\label{subsec:masking-grabcut}

GrabCut \cite{grabcut2004} operates by using the subject position in the image and some statistical inference for labeling each pixel of the image as background or foreground. 

The base algorithm \cite{opencv_grabcut} requires a bounding box as an input, a rectangle to enclose the subject to be segmented. Everything outside this rectangle will be taken as \textit{sure-background}; everything inside the rectangle is unknown. The image is first transformed into a graph, then iteratively processed using a Gaussian Mixture Model \cite{Reynolds2009} for color-based statistical labeling. A min-cut algorithm is used to segment the graph until convergence.

OpenCV GrabCut implementation has two initialization modalities. It is possible to pass only the rectangle, as explained before, or a mask of the image which further specifies whether a certain pixel is \textit{sure-background}, \textit{probable-background}, \textit{probable-foreground} or \textit{sure-foreground}. The library also uses this categorization during the algorithm itself. 

\medskip

Both approaches require previous knowledge about the subject position in the image. Our initial experiments assume that such information is given. In section \ref{subsec:masking-yolo}, we will consider an automatic human detection algorithm.


\subsubsection{Rectangle initialization}
\label{subsec:masking-grabcut-rect}

This approach requires that a rectangle, entirely containing the subject, is given in input to the function (figure \ref{fig:grabcut-rect-explain-1}). GrabCut proceeds as follows. Area inside the rectangle is marked as \textit{probable-background} (green), while the pixels outside are \textit{sure-background} (blue). As the algorithm keeps going, it finds pixels inside the rectangle which can be foreground, marking them as \textit{probable-foreground} (yellow) (figure \ref{fig:grabcut-rect-explain-2}). Later, we binarize and smooth the mask (figure \ref{fig:grabcut-rect-explain-3}) for finally removing the background from the original image.

\begin{figure}[!h]
	\begin{center}
		\begin{subfigure}[h]{0.32\textwidth}
			\centering
			\includegraphics[width=1\textwidth]{"contents/images/04-2grabcut-1rect-steps-1"}
			\caption[]{Initialization}
			\label{fig:grabcut-rect-explain-1}
		\end{subfigure}
		\hfill
		\begin{subfigure}[h]{0.32\textwidth}
			\centering
			\includegraphics[width=1\textwidth]{"contents/images/04-2grabcut-1rect-steps-2"}
			\caption[]{Pixel classes}
			\label{fig:grabcut-rect-explain-2}
		\end{subfigure}
		\hfill
		\begin{subfigure}[h]{0.32\textwidth}
			\centering
			\includegraphics[width=1\textwidth]{"contents/images/04-2grabcut-1rect-steps-3"}
			\caption[]{Final mask}
			\label{fig:grabcut-rect-explain-3}
		\end{subfigure}
	\end{center}
	\vspace{-0.5cm}
	\caption[Grabcut algorithm explained: rectangle initialization]{Grabcut algorithm explained: rectangle initialization}
	\label{fig:grabcut-rect-explain}
\end{figure}

Performance obtained by the algorithm is available in figure \ref{fig:grabcut-rect-example}. Results seem better than the ones produced by Canny Edge Detection. However, it happens that the face or the entire person is filtered out of the image.

\begin{figure}[!h]
	\begin{center}
		\begin{subfigure}[h]{1\textwidth}
			\centering
			\includegraphics[width=1\textwidth]{"contents/images/04-2grabcut-1rect-example-1"}
		\end{subfigure}
		\vfill
		\begin{subfigure}[h]{1\textwidth}
			\centering
			\includegraphics[width=1\textwidth]{"contents/images/04-2grabcut-1rect-example-2"}
		\end{subfigure}
	\end{center}
	\vspace{-0.5cm}
	\caption[Grabcut demonstration: rectangle initialization]{Grabcut demonstration: rectangle initialization}
	\label{fig:grabcut-rect-example}
\end{figure}


\subsubsection{Mask initialization}
\label{subsec:masking-grabcut-mask}

For advanced purposes, OpenCV also gives the possibility to initially classify the pixels with a mask. We choose this methodology for setting the face and part of the body as \textit{sure-foreground} from the beginning.

For demonstration, we consider an image for which the rectangle initialization above was completely missing the person in the result. Procedure is shown in figure \ref{fig:grabcut-mask-explain}. We start by specifying \textit{sure-foreground} pixels (blue), then GrabCut automatically infers \textit{probable-foreground} (yellow) and \textit{probable-background} (green) areas. Results are undoubtedly better, but we notice that the left-most background has been kept in the final image. However, this part of the image could have been easily identified as \textit{sure-background} by using the person pose we assume as known, as we did for rectangle initialization.

\begin{figure}[!h]
	\begin{center}
		\begin{subfigure}[h]{0.24\textwidth}
			\centering
			\includegraphics[width=1\textwidth]{"contents/images/04-2grabcut-2mask-steps-1"}
			\caption[]{Initialization}
			\label{fig:grabcut-mask-explain-1}
		\end{subfigure}
		\hfill
		\begin{subfigure}[h]{0.24\textwidth}
			\centering
			\includegraphics[width=1\textwidth]{"contents/images/04-2grabcut-2mask-steps-2"}
			\caption[]{Pixel classes}
			\label{fig:grabcut-mask-explain-2}
		\end{subfigure}
		\hfill
		\begin{subfigure}[h]{0.24\textwidth}
			\centering
			\includegraphics[width=1\textwidth]{"contents/images/04-2grabcut-2mask-steps-3"}
			\caption[]{Final mask}
			\label{fig:grabcut-mask-explain-3}
		\end{subfigure}
		\hfill
		\begin{subfigure}[h]{0.24\textwidth}
			\centering
			\includegraphics[width=1\textwidth]{"contents/images/04-2grabcut-2mask-steps-4"}
			\caption[]{Result}
			\label{fig:grabcut-mask-explain-4}
		\end{subfigure}
		\hfill
	\end{center}
	\vspace{-0.5cm}
	\caption[Grabcut algorithm explained: mask initialization]{Grabcut algorithm explained: mask initialization}
	\label{fig:grabcut-mask-explain}
\end{figure}


\subsubsection{Hybrid initialization}
\label{subsec:masking-grabcut-hybrid}

Finally, a mixed approach between rectangle and mask initializations has been tried. It allows to specify both the person and the face positions in the image, by initially setting \textit{sure-background}, \textit{probable-background}, and \textit{sure-foreground} pixels. Only \textit{probable-foreground} pixels have to be found by the GrabCut algorithm. 

In figure \ref{fig:grabcut-hybrid-explain} is available a step-by-step explanation like the ones presented above. In there, \textit{sure-background} is dark blue, \textit{probable-background} is green, \textit{probable-foreground} is yellow and \textit{sure-foreground} is light blue.

Image \ref{fig:grabcut-hybrid-example} shows the results of hybrid initialization applied to the same samples introduced in figure \ref{fig:grabcut-rect-example}. The segmentation is very precise, and the approach reveals to be almost optimal.

\begin{figure}[!h]
	\begin{center}
		\begin{subfigure}[h]{0.24\textwidth}
			\centering
			\includegraphics[width=1\textwidth]{"contents/images/04-2grabcut-3hybrid-steps-1"}
			\caption[]{Initialization}
			\label{fig:grabcut-hybrid-explain-1}
		\end{subfigure}
		\hfill
		\begin{subfigure}[h]{0.24\textwidth}
			\centering
			\includegraphics[width=1\textwidth]{"contents/images/04-2grabcut-3hybrid-steps-2"}
			\caption[]{Pixel classes}
			\label{fig:grabcut-hybrid-explain-2}
		\end{subfigure}
		\hfill
		\begin{subfigure}[h]{0.24\textwidth}
			\centering
			\includegraphics[width=1\textwidth]{"contents/images/04-2grabcut-3hybrid-steps-3"}
			\caption[]{Final mask}
			\label{fig:grabcut-hybrid-explain-3}
		\end{subfigure}
		\hfill
		\begin{subfigure}[h]{0.24\textwidth}
			\centering
			\includegraphics[width=1\textwidth]{"contents/images/04-2grabcut-3hybrid-steps-4"}
			\caption[]{Result}
			\label{fig:grabcut-hybrid-explain-4}
		\end{subfigure}
		\hfill
	\end{center}
	\vspace{-0.5cm}
	\caption[Grabcut algorithm explained: hybrid initialization]{Grabcut algorithm explained: hybrid initialization}
	\label{fig:grabcut-hybrid-explain}
\end{figure}

\begin{figure}[!h]
	\centering
	\includegraphics[width=1\textwidth]{"contents/images/04-2grabcut-3hybrid-final"}
	\caption[Grabcut demonstration: hybrid initialization]{Grabcut demonstration: hybrid initialization}
	\label{fig:grabcut-hybrid-example}
\end{figure}


\subsubsection{Automatic Human Detection}
\label{subsec:masking-yolo}

The approaches presented until now are used for doing human segmentation, which is an essential step for performing background removal. To work, GrabCut hybrid initialization requires two pieces of information: the bounding boxes associated with both the entire person and its head.

Given the available data, the mapping between the user's pose (the \gls{gt}) and its appearing in an image is unknown. Thus, proper detection algorithms are required on the top of the segmentation task to find both the person and its face in an image.

\medskip

For detecting the user we try YOLO \cite{redmon2016look}, a state of the art technique for object detection. We decide to adopt the \texttt{cvlib} library \cite{cvlib}, which underneath uses the OpenCV dnn module \cite{opencv_dnn} for implementing a YOLOv3 model trained on the Microsoft COCO dataset \cite{lin2015microsoft}.

A demo on our dataset is shown in figure \ref{fig:yolo}, where we notice that YOLO overall provides quite good results. However, in 10-20\% of the cases, it does not detect any object in the image, most probably because of their low resolution.

\begin{figure}[!h]
	\centering
	\includegraphics[width=1\textwidth]{"contents/images/04-3yolo"}
	\caption[YOLO demonstration, which shows failures for 2 images]{YOLO demonstration, which shows failures for 2 images}
	\label{fig:yolo}
\end{figure}

The same problem also affects face detection, which is needed for providing the related mask to the GrabCut algorithm. An open-source head detector has been tried\footnote{\url{https://github.com/AVAuco/ssd_head_keras}}. Results are poor, once again due to the small size of our images.

%\medskip

%While searching for a workaround to work with such low-fidelity images, an all-in-one satisfying solution for both human detection and segmentation has been found. The next section presents the obtained results.



\subsection{Mask R-CNN}
\label{subsec:masking-maskrcnn}

Mask R-CNN \cite{he2018mask} is a state of the art deep learning framework for object detection and instance segmentation, whose technical details have been illustrated in section \ref{subsec:sota-maskrcnn}. Initially developed by Facebook researchers in PyTorch \cite{pytorch}, the algorithm has been ported on TensorFlow 1 \cite{MaskRCNN_matterport} and later adapted for TensorFlow 2 \cite{MaskRCNN_akTwelve}\footnote{\url{https://github.com/akTwelve/Mask_RCNN}}.

\medskip

Mask R-CNN results on our dataset are incredibly precise, and the method undoubtedly outperforms any other previously experimented since it provides both human detection and segmentation at once. Figure \ref{fig:maskrcnn-dario} below presents how Mask R-CNN easily detects people in our video frames, regardless of their low resolution and any light condition or person position. In many cases, multiple people or objects in the background are correctly detected, even if they are tiny in the images.

\begin{figure}[!h]
	\begin{center}
		\begin{subfigure}[h]{0.32\textwidth}
			\centering
			\includegraphics[width=1\textwidth]{"contents/images/04-maskrcnn-train-1"}
		\end{subfigure}
		\begin{subfigure}[h]{0.32\textwidth}
			\centering
			\includegraphics[width=1\textwidth]{"contents/images/04-maskrcnn-train-2"}
		\end{subfigure}
		\begin{subfigure}[h]{0.32\textwidth}
			\centering
			\includegraphics[width=1\textwidth]{"contents/images/04-maskrcnn-train-3"}
		\end{subfigure}
		\vfill
		\begin{subfigure}[h]{0.32\textwidth}
			\centering
			\includegraphics[width=1\textwidth]{"contents/images/04-maskrcnn-train-4"}
		\end{subfigure}
		\begin{subfigure}[h]{0.32\textwidth}
			\centering
			\includegraphics[width=1\textwidth]{"contents/images/04-maskrcnn-train-5"}
		\end{subfigure}
		\begin{subfigure}[h]{0.32\textwidth}
			\centering
			\includegraphics[width=1\textwidth]{"contents/images/04-maskrcnn-train-6"}
		\end{subfigure}
	\end{center}
	\vspace{-0.5cm}
	\caption[Mask R-CNN applied to our training set]{Mask R-CNN applied to our training set}
	\label{fig:maskrcnn-dario}
\end{figure}

This high-level of accuracy in detection and segmentation comes with an extremely-high computing power requirement\footnote{according to the original paper, Mask R-CNN runs at 5 \gls{fps} on Nvidia Tesla M40 GPU}. For reference, running Mask R-CNN on the test set - composed of about 11'000 images - requires a total computing time\footnote{we observe, using the \texttt{\%\%time} command for IPython, the following CPU times: user 35min, sys 20min, total 55min; and Wall time: 1h 4min} of approximately 55 minutes on Google Colab using a GPU runtime\footnote{equipped with Nvidia T4 GPU}.

Because of this, the inference on the images must be made offline. Together with each input image, the training procedure will only receive the previously computed user's mask. This will be used to perform background replacement.

