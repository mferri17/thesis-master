\documentclass[mscthesis, 11pt, oneside, openany]{usiinfthesis}

\usepackage[utf8]{inputenc} 
\usepackage[T1]{fontenc}
\usepackage{textcomp}
\usepackage{lipsum}

\usepackage[font=small,labelfont=bf]{caption}
\usepackage{float}
\usepackage{blindtext}

\setcounter{tocdepth}{2}
\setcounter{secnumdepth}{6}

\usepackage{tocloft}
\setlength{\cftfignumwidth}{3em}

\usepackage{enumitem}

\usepackage{graphicx}
\pdfsuppresswarningpagegroup=1
\usepackage{caption}
%\usepackage[justification=justified,singlelinecheck=false]{caption}
\usepackage{subcaption}

\usepackage{url}
\makeatletter
\g@addto@macro{\UrlBreaks}{\UrlOrds}
\makeatother
\PassOptionsToPackage{hyphens}{url}
%Import the natbib package and sets a bibliographys and citation styles
%\bibliographystyle{dinat}
%\bibliographystyle{alpha}
%\bibliographystyle{dcu}
\bibliographystyle{plainnat}
\setcitestyle{authoryear,open={[},close={]}}

\newfloat{Equation}{htbp}{equ}[chapter]
\newcommand{\listequationsname}{Equations}

\usepackage{xcolor}
\usepackage{listings,lstautogobble}
\definecolor{gray}{gray}{0.5}
\colorlet{commentcolour}{green!50!black}
\colorlet{stringcolour}{red!60!black}
\colorlet{keywordcolour}{blue}
\colorlet{exceptioncolour}{yellow!50!red}
\colorlet{commandcolour}{magenta!90!black}
\colorlet{numpycolour}{blue!60!green}
\colorlet{literatecolour}{magenta!90!black}
\colorlet{promptcolour}{green!50!black}
\colorlet{specmethodcolour}{violet}
\newcommand*{\literatecolour}{\textcolor{literatecolour}}
\newcommand*{\pythonprompt}{\textcolor{promptcolour}{{>}{>}{>}}}
\lstdefinestyle{python}{
	language=python,
	showtabs=true,
	tab=,
	tabsize=4,
	basicstyle=\ttfamily\footnotesize,
	stringstyle=\color{stringcolour},
	showstringspaces=false,
	keywordstyle=\color{keywordcolour}\bfseries,
	emph={as,and,break,class,continue,def,yield,del,elif ,else,%
		except,exec,finally,for,from,global,if,in,%
		lambda,not,or,pass,print,raise,return,try,while,assert,with},
	emphstyle=\color{blue}\bfseries,
	emph={[2]True, False, None},
	emphstyle=[3]\color{commandcolour},
	morecomment=[s]{"""}{"""},
	commentstyle=\color{commentcolour}\slshape,
	emph={array, matmul, ones, transpose, float32},
	emphstyle=[4]\color{numpycolour},
	emph={[5]assert,yield},
	emphstyle=[5]\color{keywordcolour}\bfseries,
	emph={[6]range},
	emphstyle={[6]\color{keywordcolour}\bfseries},
	literate=*%
	{:}{{\literatecolour:}}{1}%
	{=}{{\literatecolour=}}{1}%
	{-}{{\literatecolour-}}{1}%
	{+}{{\literatecolour+}}{1}%
	{*}{{\literatecolour*}}{1}%
	{**}{{\literatecolour{**}}}2%
	{/}{{\literatecolour/}}{1}%
	{//}{{\literatecolour{//}}}2%
	{!}{{\literatecolour!}}{1}%
	{<}{{\literatecolour<}}{1}%
	{>}{{\literatecolour>}}{1}%
	{>>>}{\pythonprompt}{3},
	frame=trbl,
	rulecolor=\color{black!40},
	backgroundcolor=\color{gray!5},
	breakindent=.5\textwidth,
	frame=single,
	breaklines=true,
	basicstyle=\ttfamily\footnotesize,%
	keywordstyle=\color{keywordcolour},%
	emphstyle={[7]\color{keywordcolour}},%
	emphstyle=\color{exceptioncolour},%
	literate=*%
	{:}{{\literatecolour:}}{2}%
	{=}{{\literatecolour=}}{2}%
	{-}{{\literatecolour-}}{2}%
	{+}{{\literatecolour+}}{2}%
	{*}{{\literatecolour*}}2%
	{**}{{\literatecolour{**}}}3%
	{/}{{\literatecolour/}}{2}%
	{//}{{\literatecolour{//}}}{2}%
	{!}{{\literatecolour!}}{2}%
	{<}{{\literatecolour<}}{2}%
	{<=}{{\literatecolour{<=}}}3%
	{>}{{\literatecolour>}}{2}%
	{>=}{{\literatecolour{>=}}}3%
	{==}{{\literatecolour{==}}}3%
	{!=}{{\literatecolour{!=}}}3%
	{+=}{{\literatecolour{+=}}}3%
	{-=}{{\literatecolour{-=}}}3%
	{*=}{{\literatecolour{*=}}}3%
	{/=}{{\literatecolour{/=}}}3%
}
\lstnewenvironment{python}
{\lstset{style=python}}
{}

% Load the package with the acronym option
\usepackage[titletoc,page]{appendix}
\makeatletter\@openrightfalse\makeatother

\usepackage[acronym,automake,nonumberlist]{glossaries}
%\usepackage[noindex]{glossaries-extra}

% Generate the glossary
\makeglossary
\renewcommand*\glspostdescription{\dotfill}
\renewcommand*\acronymname{ }

\newcommand{\nomNoPrint}[3]{\newglossaryentry{#1}{
		name={#2},
		symbol={#2},
		description={#3},
		sort={A\three@digits{\value{page}}}
	}\glsadd[format=hyperbf]{#1}}   
\makeatother

\newcommand{\nom}[3]{\nomNoPrint{#1}{#2}{#3}#2}

\input{contents/abbreviations}
\newacronym{ai}{AI}{Artificial Intelligence}
\newacronym{ml}{ML}{Machine Learning}
\newacronym{dl}{DL}{Deep Learning}
\newacronym{dnn}{DNN}{Deep Neural Network}
\newacronym{rnn}{RNN}{Recurrent Neural Network}
\newacronym{rl}{RL}{Reinforcement Learning}
\newacronym{nn}{NN}{Neural Network}
\newacronym{cnn}{CNN}{Convolutional Neural Network}
\newacronym{ann}{ANN}{Artificial Neural Network}

\newacronym{idsia}{IDSIA}{Istituto Dalle Molle di Studi sull’Intelligenza Artificiale}
\newacronym{ros}{ROS}{Robot Operating System}

\newacronym{mse}{MSE}{Mean Squared Error}
\newacronym{rmse}{RMSE}{Rooted Mean Squared Error}
\newacronym{r2}{R$^2$}{R Squared}

\newcommand{\footlabel}[2]{%
	\addtocounter{footnote}{1}%
	\footnotetext[\thefootnote]{%
		\addtocounter{footnote}{-1}%
		\refstepcounter{footnote}\label{#1}%
		#2%
	}%
	$^{\ref{#1}}$%
}

\newcommand{\footref}[1]{%
	$^{\ref{#1}}$%
}

\lstdefinelanguage{algebra}
{morekeywords={import,sort,constructors,observers,transformers,axioms,if,
else,end},
sensitive=false,
morecomment=[l]{//s},
}


%%%%%%%%%%%%%%%%%%%%%%%%%%%%%%%%%%%%%%%%%%%%%%%%%%%%%%%%%%%%%%%%%%%%%%%%%%%%%%%%%%%%%%%%%%%%%%%%%%%%%%%%%%%%%%


\title{Interpretation of neural networks and \\ advanced image augmentation for visual end-to-end control of drones} %compulsory
%\specialization{Dependable Distributed Systems}%optional
%\subtitle{Subtitle: Reinventing the World} %optional 

\author{Marco Ferri} %compulsory
\begin{committee}
\advisor{Prof.}{Alessandro}{Giusti} %compulsory
\coadvisor{Dr.}{Dario}{Mantegazza}{} %optional
\end{committee}

\Day{22} %compulsory
\Month{February} %compulsory
\Year{2021} %compulsory, put only the year
\place{Lugano} %compulsory

\dedication{To someone} %optional
\openepigraph{``Sometimes it is the people no one can imagine anything of, who do the things no one can imagine.''}{The Imitation Game} %optional


%\makeindex %optional, also comment out \theindex at the end


%%%%%%%%%%%%%%%%%%%%%%%%%%%%%%%%%%%%%%%%%%%%%%%%%%%%%%%%%%%%%%%%%%%%%%%%%%%%%%%%%%%%%%%%%%%%%%%%%%%%%%%%%%%%%%

\begin{document}
	
	
\maketitle %generates the titlepage, this is FIXED
\let\cleardoublepage\clearpage

\frontmatter %generates the frontmatter, this is FIXED
%\let\cleardoublepage\clearpage
\begingroup
%\let\cleardoublepage\clearpage


\begin{abstract}
\addcontentsline{toc}{chapter}{Abstract}  % added 
Lorem ipsum dolor sit amet, consectetur adipiscing elit. Aliquam vulputate erat quis justo varius vehicula. Vestibulum ante ipsum primis in faucibus orci luctus et ultrices posuere cubilia curae; In ut placerat velit. Pellentesque habitant morbi tristique senectus et netus et malesuada fames ac turpis egestas. In elementum egestas turpis et auctor. Vestibulum gravida lorem nec egestas ornare. Duis varius arcu imperdiet, feugiat odio in, facilisis est. Quisque interdum vitae odio ut vehicula. Etiam molestie enim non risus maximus, vitae efficitur mauris sollicitudin. Phasellus consequat nulla at nulla tempus varius. Lorem ipsum dolor sit amet, consectetur adipiscing elit.

Nulla faucibus aliquam nisl, vel luctus arcu semper vel. Aliquam ipsum risus, feugiat quis nulla ac, aliquet imperdiet ante. Ut et massa sem. Donec eu ex augue. Nam urna nunc, commodo ac nunc et, auctor mattis nunc. Nam pellentesque laoreet purus, a tristique nisi auctor non. Integer sed congue lorem. Maecenas faucibus turpis nec ultrices tempor. Vivamus condimentum nibh sit amet molestie tempor. Pellentesque cursus diam maximus nisi gravida, sed consequat mauris malesuada. Fusce eget nisl vehicula, porta enim sit amet, ornare massa. Nunc et ex a eros tempor mattis in quis quam. Lorem ipsum dolor sit amet, consectetur adipiscing elit. Duis volutpat ex nec ante tempus, quis bibendum nisi posuere. Cras augue nulla, ornare vel risus quis, vulputate bibendum sem.

Proin placerat euismod cursus. Nulla ornare lobortis ligula et pellentesque. Nullam cursus neque ut fermentum euismod. Sed sit amet luctus orci. Nunc convallis urna id nisl vestibulum ullamcorper. Aliquam ullamcorper porta dui et aliquam. Pellentesque urna nibh, finibus sed condimentum eget, interdum ut tellus. Morbi aliquet, erat et rhoncus cursus, lectus nibh vulputate nisl, eu interdum dui dolor quis ipsum. Donec id libero sit amet orci gravida pretium. Mauris in magna non nunc posuere consequat. Donec diam tortor, viverra posuere velit et, convallis commodo massa. Fusce consectetur posuere ex, nec tincidunt neque posuere tempus. Vivamus vitae accumsan ligula. Nulla facilisi. Donec pellentesque commodo lorem ac semper.
\end{abstract}


\begin{acknowledgements}
%\addcontentsline{toc}{chapter}{Acknowledgements}  % added 

My most grateful thanks go to my supervisors Alessandro and Dario for having thoroughly assisted me, both technically and emotionally, during the entire development of this thesis.

\smallskip

Thanks to the University of Milano-Bicocca for the opportunity of participating in this amazing Double Degree Program and to USI Università della Svizzera Italiana for immediately making me feel at home. Thanks to all my classmates and friends, who have lightened my journey over all these years.

\bigskip

A special thanks to my lovely Nadia, for always being by my side, supporting me unconditionally, and making me laugh with every single breath.

\smallskip

Un ultimo e sentito grazie ai miei genitori, per aver sempre appoggiato le mie scelte e per avermi trasmesso i loro valori, rendendomi la persona che sono oggi.
\end{acknowledgements}
\endgroup


\tableofcontents 
\newpage

\begingroup
\let\cleardoublepage\clearpage
\listoffigures %optional
\addcontentsline{toc}{chapter}{List of Figures}

\clearpage%\cleardoublepage
\listoftables %optional
\addcontentsline{toc}{chapter}{List of Tables}

%%\cleardoublepage
%\listof{Equation}{\listequationsname}
%\addcontentsline{toc}{chapter}{List of \listequationsname}

%\lstlistoflistings
%\addcontentsline{toc}{chapter}{List of Listings}
%\endgroup
%%\let\cleardoublepage\relax

\makeatletter
\renewcommand\mainmatter{\clearpage\@mainmattertrue\pagenumbering{arabic}}
\makeatother

%%%%%%%%%%%%%%%%%%%%%%%%%%%%%%%%%%%%%%%%%%%

\mainmatter

\begingroup
\let\cleardoublepage\clearpage
\chapter{Introduction}
%\addcontentsline{toc}{chapter}{Introduction}
%\markboth{}{}
\label{chap:intro}

\glsreset{ai}
\glsreset{dl}


% This project was developed at Istituto Dalle Molle di Studi sull'Intelligenza Artifciale
% (IDSIA) laboratories in Manno, Switzerland for the Master Thesis inside a double degree
% program between Universities of Milano Bicocca and USI, Universitò della Svizzera Ital-
% iana of Lugano.
% Our entire work and additional details are available at:
% https://github.com/mferri17/cnn-drone-befree.



\section{Objective}
\label{sec:objective}

Autonomous navigation is an important and well-known research area in robotics, which usually requires accomplishing complex and computationally-expensive tasks such as localization, mapping, and path planning. Recent studies have started to approach autonomous driving through imitation learning \cite{imitation_learning_survey}. The approach consist in making a neural network learn its task by imitating human or robot behaviors.

Lorem ipsum dolor sit amet.




\section{Outline}
\label{sec:outline}

The thesis is composed of X chapters:
\begin{itemize}
	\item Chapter summarises ;
	
	\item Chapter provides ;
	
	\item Chapter presents ;
		
	\item Chapter illustrates ;

	\item Chapter explores ;	
	
	\item Chapter addresses .
\end{itemize}






\chapter{Conclusion and perspectives}
%\addcontentsline{toc}{chapter}{Conclusion and perspectives}
\label{chap:concl}
 
 
This chapter presents first, in Section \ref{sec:concl}, our concluding thoughts and 
finally suggestions for possible future research lines in Section \ref{sec:future}.



\section{Concluding thoughts}
\label{sec:concl}

\lipsum[2]



\section{Future works}
\label{sec:future}

\lipsum[2]



%\setcounter{chapter}{0}
\chapter{Latex features}
\label{chap:test}

\glsreset{ai}
\glsreset{dl}

\begin{verbatim}
ACCURACY    10-FOLD CROSS VALIDATION:    0.8290 (std dev 0.005217)
PRECISION   10-FOLD CROSS VALIDATION:    0.8371 (std dev 0.009706)
RECALL      10-FOLD CROSS VALIDATION:    0.8176 (std dev 0.008412)
F1          10-FOLD CROSS VALIDATION:    0.8273 (std dev 0.004608)
\end{verbatim}

In this work we consider cooperative multi-agent scenarios, in which multiple 
robots collaborate, and possibly communicate, to achieve a common goal 
\cite[][]{ismail2018survey}. Let'try this \footnote{this is a footnote}.
Homogeneous Multi-Agent Systems (MAS) are composed of $N$ interacting 
agents, which have the same physical structure and observation capabilities, so 
they can be considered to be interchangeable and cooperate to solve a given 
task \cite[][]{stone2000multiagent, vsovsic2016inverse}.
This system is characterised by a state $S$ — which can be decomposed in sets of 
local states for each agent and the set of possible observations $O$ for each agent 
— obtained through sensors and the set of possible actions  $A$ for each agent.

\begin{itemize}
	\item prodotti e utenti nella stessa rete, collegati attraverso le recensioni
	\item rete di soli prodotti, collegati tramite similarità
	\item rete di soli utenti, collegati tramite similarità
\end{itemize}

\begin{figure}[!htb]
	\begin{center}
		\begin{subfigure}[h]{0.30\textwidth}
			\centering
			\includegraphics[width=\textwidth]{"contents/images/20210110_172800 - origtest11030(len11030)_fps30_w2"}
		\end{subfigure}
		\hfill
		\begin{subfigure}[h]{0.68\textwidth}
			\centering
			\includegraphics[width=\textwidth]{"contents/images/20210110_172810 - origtest11030(len11030)_fps30_w5"}
		\end{subfigure}
	\end{center}
	\vspace{-0.5cm}
	\caption[list of figures caption]{current caption}
	\label{fig:task1}
	%	\vspace{-0.5cm}
\end{figure}

\lipsum[2]

\bigskip

\begin{figure}[!htb]
	\centering
	\includegraphics[width=.55\textwidth]{"contents/images/20210110_172810 - origtest11030(len11030)_fps30_w5"}
	\caption[list of figures caption]{current caption \cite[][]{shojaei2011adaptive}.}
	\label{fig:differentialdrive}
\end{figure}

\lipsum[2]

\bigskip

\begin{python}

def tfdata_generator(files, input_size, batch_size, backgrounds, 
					 bg_smoothmask, aug_prob=0, noises=[],
					 prefetch=True, parallelize=True, 
					 deterministic=False, cache=False, repeat=1):

	map_parallel = tf.data.experimental.AUTOTUNE if parallelize else None
	backgrounds = tf.convert_to_tensor(backgrounds) # saves time during training
	noises = tf.convert_to_tensor(noises) # saves time during training

	gen = tf.data.Dataset.from_tensor_slices(files)
	gen = gen.map(lambda filename: map_parse_input(filename, input_size), map_parallel, deterministic)
	
	if cache:
		gen = gen.cache()
		if not deterministic: # shuffling would destroy determinism
			gen = gen.shuffle(len(files), reshuffle_each_iteration=True)
	
	gen = gen.map(lambda img, mask, gt: map_replace_background(img, mask, gt, backgrounds, bg_smoothmask), map_parallel, deterministic)
	gen = gen.map(lambda img, gt: map_augmentation(img, gt, aug_prob, noises), map_parallel, deterministic)
	gen = gen.map(lambda img, gt: map_preprocessing(img, gt), map_parallel, deterministic)
	gen = gen.batch(batch_size, drop_remainder=True)
	gen = gen.repeat(repeat)
	
	if prefetch:
		gen = gen.prefetch(tf.data.experimental.AUTOTUNE)
	
	return gen
\end{python}

\begin{lstlisting}[frame=none,caption={Protocol used by the manual controller 
to decide, for each robot, the message to transmit and the colour.}, 
label=lst:manualtask2]
\end{lstlisting}


\begin{table}[H]
	\caption{Schema originale del dataset}\label{tab:df-schema}
	\centering
	\begin{tabular}{|l|l|l|}
		\hline
		Campo & Descrizione \\
		\hline
		reviewerID & ID utente \\
		reviewerName & Nome utente \\
		asin & ID prodotto \\
		reviewText & Testo della recensione \\
		summary & Titolo della recensione \\
		helpful & Utilità della recensione \\
		overall & Punteggio \\
		reviewTime & Timestamp in formato string \\
		unixReviewTime & Timestamp in formato unix \\
		\hline
	\end{tabular}
\end{table}


\paragraph*{\texttt{Numpy}}
\lipsum[1]


\paragraph*{\texttt{Keras}}
\lipsum[1]
\endgroup

%%%%%%%%%%%%%%%%%%%%%%%%%%%%%%%%%%%%%%%%%%%

\appendix %optional, use only if you have an appendix

%\begingroup
%\chapter{List of abbreviations}
%\label{chp:abbreviations}
%\let\clearpage\relax
%\vspace*{-90pt}
%\printglossary[title= ]
%\endgroup

\begingroup
\chapter{List of Acronyms}
\label{chp:acronyms}
\let\clearpage\relax
%\glsaddall  % comment to hide unused acronyms
\vspace*{-120pt}
\printglossary[type=acronym]
\endgroup

\backmatter

\nocite{*}
\bibliography{biblio}
%\cleardoublepage
%\theindex %optional, use only if you have an index, must use
%\makeindex in the preamble


\end{document}

%%%%%%%%%%%%%%%%%%%%%%%%%%%%%%%%%%%%%%%%%%%%%%%%%%%%%%%%%%%%%%%%%%%%%%%%%%%%%%%%%%%%%%%%%%%%%%%%%%%%%%%%%%%%%%
