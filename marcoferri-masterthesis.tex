\documentclass[mscthesis, 11pt, oneside, openany]{usiinfthesis}

\usepackage[utf8]{inputenc} 
\usepackage[T1]{fontenc}
\usepackage{textcomp}
\usepackage{lipsum}

\usepackage[font=small,labelfont=bf]{caption}
\usepackage{float}
\usepackage{blindtext}

\setcounter{tocdepth}{2}
\setcounter{secnumdepth}{6}

\usepackage{tocloft}
\setlength{\cftfignumwidth}{3em}

\usepackage{enumitem}

\usepackage{graphicx}
\pdfsuppresswarningpagegroup=1
\usepackage{caption}
%\usepackage[justification=justified,singlelinecheck=false]{caption}
\usepackage{subcaption}

\usepackage{url}
\makeatletter
\g@addto@macro{\UrlBreaks}{\UrlOrds}
\makeatother
\PassOptionsToPackage{hyphens}{url}
%Import the natbib package and sets a bibliographys and citation styles
%\bibliographystyle{dinat}
%\bibliographystyle{alpha}
%\bibliographystyle{dcu}
\bibliographystyle{plainnat}
\setcitestyle{authoryear,open={[},close={]}}

\newfloat{Equation}{htbp}{equ}[chapter]
\newcommand{\listequationsname}{Equations}

\usepackage{xcolor}
\usepackage{listings,lstautogobble}
\definecolor{gray}{gray}{0.5}
\colorlet{commentcolour}{green!50!black}
\colorlet{stringcolour}{red!60!black}
\colorlet{keywordcolour}{blue}
\colorlet{exceptioncolour}{yellow!50!red}
\colorlet{commandcolour}{magenta!90!black}
\colorlet{numpycolour}{blue!60!green}
\colorlet{literatecolour}{magenta!90!black}
\colorlet{promptcolour}{green!50!black}
\colorlet{specmethodcolour}{violet}
\newcommand*{\literatecolour}{\textcolor{literatecolour}}
\newcommand*{\pythonprompt}{\textcolor{promptcolour}{{>}{>}{>}}}
\lstdefinestyle{python}{
	language=python,
	showtabs=true,
	tab=,
	tabsize=4,
	basicstyle=\ttfamily\footnotesize,
	stringstyle=\color{stringcolour},
	showstringspaces=false,
	keywordstyle=\color{keywordcolour}\bfseries,
	emph={as,and,break,class,continue,def,yield,del,elif ,else,%
		except,exec,finally,for,from,global,if,in,%
		lambda,not,or,pass,print,raise,return,try,while,assert,with},
	emphstyle=\color{blue}\bfseries,
	emph={[2]True, False, None},
	emphstyle=[3]\color{commandcolour},
	morecomment=[s]{"""}{"""},
	commentstyle=\color{commentcolour}\slshape,
	emph={array, matmul, ones, transpose, float32},
	emphstyle=[4]\color{numpycolour},
	emph={[5]assert,yield},
	emphstyle=[5]\color{keywordcolour}\bfseries,
	emph={[6]range},
	emphstyle={[6]\color{keywordcolour}\bfseries},
	literate=*%
	{:}{{\literatecolour:}}{1}%
	{=}{{\literatecolour=}}{1}%
	{-}{{\literatecolour-}}{1}%
	{+}{{\literatecolour+}}{1}%
	{*}{{\literatecolour*}}{1}%
	{**}{{\literatecolour{**}}}2%
	{/}{{\literatecolour/}}{1}%
	{//}{{\literatecolour{//}}}2%
	{!}{{\literatecolour!}}{1}%
	{<}{{\literatecolour<}}{1}%
	{>}{{\literatecolour>}}{1}%
	{>>>}{\pythonprompt}{3},
	frame=trbl,
	rulecolor=\color{black!40},
	backgroundcolor=\color{gray!5},
	breakindent=.5\textwidth,
	frame=single,
	breaklines=true,
	basicstyle=\ttfamily\footnotesize,%
	keywordstyle=\color{keywordcolour},%
	emphstyle={[7]\color{keywordcolour}},%
	emphstyle=\color{exceptioncolour},%
	literate=*%
	{:}{{\literatecolour:}}{2}%
	{=}{{\literatecolour=}}{2}%
	{-}{{\literatecolour-}}{2}%
	{+}{{\literatecolour+}}{2}%
	{*}{{\literatecolour*}}2%
	{**}{{\literatecolour{**}}}3%
	{/}{{\literatecolour/}}{2}%
	{//}{{\literatecolour{//}}}{2}%
	{!}{{\literatecolour!}}{2}%
	{<}{{\literatecolour<}}{2}%
	{<=}{{\literatecolour{<=}}}3%
	{>}{{\literatecolour>}}{2}%
	{>=}{{\literatecolour{>=}}}3%
	{==}{{\literatecolour{==}}}3%
	{!=}{{\literatecolour{!=}}}3%
	{+=}{{\literatecolour{+=}}}3%
	{-=}{{\literatecolour{-=}}}3%
	{*=}{{\literatecolour{*=}}}3%
	{/=}{{\literatecolour{/=}}}3%
}
\lstnewenvironment{python}
{\lstset{style=python}}
{}

% Load the package with the acronym option
\usepackage[titletoc,page]{appendix}
\makeatletter\@openrightfalse\makeatother

\usepackage[acronym,automake,nonumberlist]{glossaries}
%\usepackage[noindex]{glossaries-extra}

% Generate the glossary
\makeglossary
\renewcommand*\glspostdescription{\dotfill}
\renewcommand*\acronymname{ }

\newcommand{\nomNoPrint}[3]{\newglossaryentry{#1}{
		name={#2},
		symbol={#2},
		description={#3},
		sort={A\three@digits{\value{page}}}
	}\glsadd[format=hyperbf]{#1}}   
\makeatother

\newcommand{\nom}[3]{\nomNoPrint{#1}{#2}{#3}#2}

\newglossaryentry{fig}{
	name={Fig.},
	description={Figure}
}


\newglossaryentry{tab}{
	name={Tab.},
	description={Table}
}

\newglossaryentry{cm}{
	name={cm},
	description={centimetre}
}

\newglossaryentry{s}{
	name={s},
	description={seconds}
}


\newglossaryentry{h}{
	name={h},
	description={hours}
}

\newglossaryentry{cm/s}{
	name={cm/s},
	description={centimeters per second}
}

\newglossaryentry{wrt}{name={w.r.t.},description={with respect to}}

\newacronym{idsia}{IDSIA}{Istituto Dalle Molle di Studi sull’Intelligenza Artificiale}
\newacronym{faa}{FAA}{United States Federal Aviation Administration}
\newacronym{ros}{ROS}{Robot Operating System}
\newacronym{vr}{VR}{Virtual Reality}
\newacronym{ar}{AR}{Augmented Reality}

\newacronym{fps}{FPS}{frames per second}
\newacronym{mp}{MP}{megapixel}
\newacronym{fov}{FOV}{field of view}
\newacronym{mocap}{MoCap}{motion capture}
\newacronym{dof}{DoF}{degrees of freedom}
\newacronym{ir}{IR}{infrared}

\newacronym{wrt}{wrt}{with respect to}

\newacronym{ai}{AI}{Artificial Intelligence}
\newacronym{xai}{XAI}{Explainable AI}
\newacronym{ml}{ML}{Machine Learning}
\newacronym{dl}{DL}{Deep Learning}

\newacronym{nn}{NN}{Neural Network}
\newacronym{dnn}{DNN}{Deep Neural Network}
\newacronym{ann}{ANN}{Artificial Neural Network}
\newacronym{rnn}{RNN}{Recurrent Neural Network}
\newacronym{rl}{RL}{Reinforcement Learning}
\newacronym{cnn}{CNN}{Convolutional Neural Network}
\newacronym{resnet}{ResNet}{Residual Neural Network}

\newacronym{cam}{CAM}{Class Activation Mapping}
\newacronym{gradcam}{Grad-CAM}{Gradient-weighted Class Activation Mapping}
\newacronym{gmm}{GMM}{Gaussian Mixture Model}

\newacronym{gt}{GT}{ground truth}
\newacronym{adam}{ADAM}{Adaptive Moment Estimation}
\newacronym{mae}{MAE}{Mean Absolute Error}
\newacronym{mse}{MSE}{Mean Squared Error}
\newacronym{rmse}{RMSE}{Rooted Mean Squared Error}
\newacronym{r2}{R$^2$}{R Squared}

\newcommand{\footlabel}[2]{%
	\addtocounter{footnote}{1}%
	\footnotetext[\thefootnote]{%
		\addtocounter{footnote}{-1}%
		\refstepcounter{footnote}\label{#1}%
		#2%
	}%
	$^{\ref{#1}}$%
}

\newcommand{\footref}[1]{%
	$^{\ref{#1}}$%
}

\lstdefinelanguage{algebra}
{morekeywords={import,sort,constructors,observers,transformers,axioms,if,
else,end},
sensitive=false,
morecomment=[l]{//s},
}


%%%%%%%%%%%%%%%%%%%%%%%%%%%%%%%%%%%%%%%%%%%%%%%%%%%%%%%%%%%%%%%%%%%%%%%%%%%%%%%%%%%%%%%%%%%%%%%%%%%%%%%%%%%%%%


\title{Interpretation of neural networks and \\ advanced image augmentation for visual end-to-end control of drones} %compulsory
%\specialization{Dependable Distributed Systems}%optional
%\subtitle{Subtitle: Reinventing the World} %optional 

\author{Marco Ferri} %compulsory
\begin{committee}
\advisor{Prof.}{Alessandro}{Giusti} %compulsory
\coadvisor{Dr.}{Dario}{Mantegazza}{} %optional
\end{committee}

\Day{22} %compulsory
\Month{February} %compulsory
\Year{2021} %compulsory, put only the year
\place{Lugano} %compulsory

\dedication{To someone} %optional
\openepigraph{``Sometimes it is the people no one can imagine anything of, who do the things no one can imagine.''}{The Imitation Game} %optional


%\makeindex %optional, also comment out \theindex at the end


%%%%%%%%%%%%%%%%%%%%%%%%%%%%%%%%%%%%%%%%%%%%%%%%%%%%%%%%%%%%%%%%%%%%%%%%%%%%%%%%%%%%%%%%%%%%%%%%%%%%%%%%%%%%%%

\begin{document}
	
	
\maketitle %generates the titlepage, this is FIXED
\let\cleardoublepage\clearpage

\frontmatter %generates the frontmatter, this is FIXED
%\let\cleardoublepage\clearpage
\begingroup
%\let\cleardoublepage\clearpage


\begin{abstract}
\addcontentsline{toc}{chapter}{Abstract}  % added 
Lorem ipsum dolor sit amet, consectetur adipiscing elit. Aliquam vulputate erat quis justo varius vehicula. Vestibulum ante ipsum primis in faucibus orci luctus et ultrices posuere cubilia curae; In ut placerat velit. Pellentesque habitant morbi tristique senectus et netus et malesuada fames ac turpis egestas. In elementum egestas turpis et auctor. Vestibulum gravida lorem nec egestas ornare. Duis varius arcu imperdiet, feugiat odio in, facilisis est. Quisque interdum vitae odio ut vehicula. Etiam molestie enim non risus maximus, vitae efficitur mauris sollicitudin. Phasellus consequat nulla at nulla tempus varius. Lorem ipsum dolor sit amet, consectetur adipiscing elit.

Nulla faucibus aliquam nisl, vel luctus arcu semper vel. Aliquam ipsum risus, feugiat quis nulla ac, aliquet imperdiet ante. Ut et massa sem. Donec eu ex augue. Nam urna nunc, commodo ac nunc et, auctor mattis nunc. Nam pellentesque laoreet purus, a tristique nisi auctor non. Integer sed congue lorem. Maecenas faucibus turpis nec ultrices tempor. Vivamus condimentum nibh sit amet molestie tempor. Pellentesque cursus diam maximus nisi gravida, sed consequat mauris malesuada. Fusce eget nisl vehicula, porta enim sit amet, ornare massa. Nunc et ex a eros tempor mattis in quis quam. Lorem ipsum dolor sit amet, consectetur adipiscing elit. Duis volutpat ex nec ante tempus, quis bibendum nisi posuere. Cras augue nulla, ornare vel risus quis, vulputate bibendum sem.

Proin placerat euismod cursus. Nulla ornare lobortis ligula et pellentesque. Nullam cursus neque ut fermentum euismod. Sed sit amet luctus orci. Nunc convallis urna id nisl vestibulum ullamcorper. Aliquam ullamcorper porta dui et aliquam. Pellentesque urna nibh, finibus sed condimentum eget, interdum ut tellus. Morbi aliquet, erat et rhoncus cursus, lectus nibh vulputate nisl, eu interdum dui dolor quis ipsum. Donec id libero sit amet orci gravida pretium. Mauris in magna non nunc posuere consequat. Donec diam tortor, viverra posuere velit et, convallis commodo massa. Fusce consectetur posuere ex, nec tincidunt neque posuere tempus. Vivamus vitae accumsan ligula. Nulla facilisi. Donec pellentesque commodo lorem ac semper.
\end{abstract}


\begin{acknowledgements}
%\addcontentsline{toc}{chapter}{Acknowledgements}  % added 

My most grateful thanks go to my supervisors Alessandro and Dario for having thoroughly assisted me, both technically and emotionally, during the entire development of this thesis.

\smallskip

Thanks to the University of Milano-Bicocca for the opportunity of participating in this amazing Double Degree Program and to USI Università della Svizzera Italiana for immediately making me feel at home. Thanks to all my classmates and friends, who have lightened my journey over all these years.

\bigskip

A special thanks to my lovely Nadia, for always being by my side, supporting me unconditionally, and making me laugh with every single breath.

\smallskip

Un ultimo e sentito grazie ai miei genitori, per aver sempre appoggiato le mie scelte e per avermi trasmesso i loro valori, rendendomi la persona che sono oggi.
\end{acknowledgements}
\endgroup


\tableofcontents 
\newpage

\begingroup
\let\cleardoublepage\clearpage
\listoffigures %optional
\addcontentsline{toc}{chapter}{List of Figures}

\clearpage%\cleardoublepage
\listoftables %optional
\addcontentsline{toc}{chapter}{List of Tables}

%%\cleardoublepage
%\listof{Equation}{\listequationsname}
%\addcontentsline{toc}{chapter}{List of \listequationsname}

%\lstlistoflistings
%\addcontentsline{toc}{chapter}{List of Listings}
%\endgroup
%%\let\cleardoublepage\relax

\makeatletter
\renewcommand\mainmatter{\clearpage\@mainmattertrue\pagenumbering{arabic}}
\makeatother

%%%%%%%%%%%%%%%%%%%%%%%%%%%%%%%%%%%%%%%%%%%

\mainmatter

\begingroup
\let\cleardoublepage\clearpage
\chapter{Introduction}
%\addcontentsline{toc}{chapter}{Introduction}
%\markboth{}{}
\label{chap:intro}

\glsresetall

Nowadays, the use of \gls{ai} is increasing in Robotics and Computer Vision. Research in the area aims to find innovative solutions, especially using \gls{dl}, for well-known yet complex goals such as autonomous navigation, human-robot interaction, and object detection. A trending approach in the last years is Imitation Learning \cite{imitation_learning_survey}. It consists of training a \gls{nn} on a given task by observing the behavior of experienced agents in the same job.

Being a branch of \gls{dl}, also \gls{il} requires a large amount of data to train on. A fundamental aspect in research is the collection of real-world data to build datasets for \gls{ml} applications. However, in many cases, data is not easy to retrieve. In Robotics, datasets acquired with physical robots are usually limited in size or do not provide a faithful representation of the real world. This happens because many Robotics applications require the robots to act in a well-controlled environment, due to physical, technical, or legal constraints. This is why, in order to provide enough training data to the \gls{ml} model, Imitation Learning often relies on datasets generated in simulation. If the simulator is good enough in proposing the neural network a good variety of realistic data, then the real world may appear to the model as just another variation of the training environment.  The technique is known as Domain Randomization \cite{tobin2017domain}, designed to transfer a learned task from one domain to another.

\medskip

In this thesis, we consider an Imitation Learning technique for teaching a drone how to continuously hover in front of a moving person. We focus on the existing work from Mantegazza et al. \cite{mantegazza2019visionbased}, aiming to improve the generalization capabilities of the model. In the paper, the authors propose a reactive control procedure for controlling the quadcopter using the \gls{ml}-inferred user's pose. They build a \gls{resnet} to predict the user's 3D coordinates with respect to the drone, using in input the images coming from an on-board camera. The network is modeled to perform regression through supervised learning, in which the ground truth is given by an external \gls{mocap} system.

The original approach is an interesting starting point for many other Robotics applications, as it provides a small but fast neural network capable of producing real-time inferences. Also, it explores a challenging task in the research area of human-drone interaction, which will most probably experience interesting growth in the next years \cite{human-drone-sota}. The main issue of the proposed model is the inability of making proper predictions outside of the training room. Since the data collection requires a dedicated \gls{mocap} system to acquire the ground truth, the possibility of recording new data in different locations must be excluded. A related work \cite{zimmerman2020thesis} explored different kinds of data augmentation for the task, applying classical image transformations. Results are encouraging but still not enough to make the model able to generalize the task in unknown environments.

\medskip

Our goal is to understand the underlying causes of the issue and find a solution to the generalization problem. First, we apply \gls{gradcam} \cite{Selvaraju_2019} for network interpretation. In a few words, the algorithm produces a heatmap on the given images which enables the visualization of the input regions actually responsible for predicting a certain model’s output. Using \gls{gradcam}, we discover that the frontal drone model is not only considering the people in the input images for producing its outputs. Instead, many recurrent elements in the training set are attracting the network's attention. We want to reduce the model overfitting by eliminating the biases coming from the data. 

As a solution, we propose an innovative pipeline for image augmentation, inspired by Domain Randomization \cite{weng2019DR}. Our approach consists of modifying the original dataset through the replacement of the camera's frames' background. We rely on Mask R-CNN \cite{he2018mask}, a state of the art deep learning algorithm for object detection and segmentation. We adopt the algorithm to pre-compute the users' masks, used during training to blend the input frames with other images, serving as the backgrounds. For the replacement, we use images from a public dataset for Indoor Scene Recognition \cite{cvpr09}. The technique allows the simulation of various scenarios without the need to actually collecting new data. We also apply classic image augmentation before retraining the \gls{resnet} architecture on the background-replaced dataset. Several quantitative and qualitative experiments demonstrate the robustness of the solution, providing satisfactory results in a large variety of real-world scenarios, both indoor and outdoor.

\medskip

The solution we propose is reasonably applicable to other Computer Vision tasks. In particular, the usage of a computationally expensive method such as Mask R-CNN for pre-processing data enables a severe expansion of a given dataset. Hence, the augmented dataset can be used to train lighter models, which may require human or object recognition capabilities at a cheaper cost.

\clearpage

This work is the result of a collaboration with the Robotics research team at \gls{idsia}, in Lugano (Switzerland). The thesis has been submitted to the \gls{unimib} and \gls{usi} in the context of a Double Degree Program for the Master of Science in Informatics.

The entire source code and any additional resources, presentations, or videos are publicly available at \url{https://github.com/mferri17/cnn-drone-befree}.




\section*{Thesis Outline}
\label{sec:outline}

The document is composed of seven chapters, briefly described here:

\begin{itemize}
	\item This chapter provides an overview of our work and its structure.
	\item Chapter \ref{chap:theory} gives a theoretical introduction on the concepts used in the thesis and explores the main literature related with the thesis.
	\item Chapter \ref{chap:system} illustrates the composition of the existing system and lists the main frameworks used during the development of our software.
	\item Chapter \ref{chap:design} summarizes our design process and choices by presenting the experiments conducted for shaping our final solution.
	\item Chapter \ref{chap:implementation} carefully describes practical details of our implementation from both methodology and technical perspectives.
	\item Chapter \ref{chap:evaluation} shows the evaluation results for assessing our approach validity and robustness from a quantitative and a qualitative point of view.
	\item Chapter \ref{chap:conclusion} concludes the thesis, summarizing what has been done with some final thoughts and future works.
\end{itemize}

Appendix \ref{chap:extra-figures} also contains additional images, not inserted in the main chapters, that add minor interesting details on various topics.






\chapter{Conclusion}
%\addcontentsline{toc}{chapter}{Conclusion and perspectives}
\label{chap:conclusion}
 
 
This chapter presents first, in Section \ref{sec:concl}, our concluding thoughts and 
finally suggestions for possible future research lines in Section \ref{sec:future}.



\section{Final Thoughts}
\label{sec:concl}

\lipsum[2]



\section{Future Works}
\label{sec:future}

\lipsum[2]



%\setcounter{chapter}{0}
\chapter{Test features}
\label{chap:test}

\glsreset{ai}
\glsreset{dl}

In this work we consider cooperative multi-agent scenarios, in which multiple 
robots collaborate, and possibly communicate, to achieve a common goal 
\cite[][]{ismail2018survey}. Let'try this \footnote{this is a footnote}.
Homogeneous Multi-Agent Systems (MAS) are composed of $N$ interacting 
agents, which have the same physical structure and observation capabilities, so 
they can be considered to be interchangeable and cooperate to solve a given 
task \cite[][]{stone2000multiagent, vsovsic2016inverse}.
This system is characterised by a state $S$ — which can be decomposed in sets of 
local states for each agent and the set of possible observations $O$ for each agent 
— obtained through sensors and the set of possible actions  $A$ for each agent.

\begin{figure}[!htb]
	\begin{center}
		\begin{subfigure}[h]{0.49\textwidth}
			\centering
			\includegraphics[width=\textwidth]{"contents/images/20210110_172800 - origtest11030(len11030)_fps30_w2"}
		\end{subfigure}
		\hfill
		\begin{subfigure}[h]{0.49\textwidth}
			\centering
			\includegraphics[width=\textwidth]{"contents/images/20210110_172810 - origtest11030(len11030)_fps30_w5"}
		\end{subfigure}
	\end{center}
	\vspace{-0.5cm}
	\caption[Visualisation of the simulation of the first task.]{Visualisation of the 
		initial and final configurations obtained simulating the first task.}
	\label{fig:task1}
	%	\vspace{-0.5cm}
\end{figure}

\lipsum[2]

\bigskip

\begin{figure}[!htb]
	\centering
	\includegraphics[width=.55\textwidth]{"contents/images/20210110_172810 - origtest11030(len11030)_fps30_w5"}
	\caption[Non-holonomic differential drive mobile robot.]{Configuration of a 
		non-holonomic differential drive mobile robot 
		\cite[][]{shojaei2011adaptive}.}
	\label{fig:differentialdrive}
\end{figure}

\lipsum[2]

\bigskip

\begin{python}
c_left, c_right = get_received_communication(state)

if N % 2 == 1:  # if the number of robots is odd

# Case 1: no communication received from left
if c_left == 0:
if c_right > N // 2:
# the agent is in the first half of the row, so its colour is blue
message = c_right - 1
colour = 1
elif c_right == N // 2:
# the agent is the central one, so its colour is blue
message = c_right + 1
colour = 1
\end{python}
\begin{lstlisting}[frame=none,caption={Protocol used by the manual controller 
to decide, for each robot, the message to transmit and the colour.}, 
label=lst:manualtask2]
\end{lstlisting}


\paragraph*{\texttt{Numpy}}
\lipsum[1]


\paragraph*{\texttt{Keras}}
\lipsum[1]
\endgroup

%%%%%%%%%%%%%%%%%%%%%%%%%%%%%%%%%%%%%%%%%%%

\appendix %optional, use only if you have an appendix

%\begingroup
%\chapter{List of abbreviations}
%\label{chp:abbreviations}
%\let\clearpage\relax
%\vspace*{-90pt}
%\printglossary[title= ]
%\endgroup

\begingroup
\chapter{List of Acronyms}
\label{chp:acronyms}
\let\clearpage\relax
%\glsaddall  % comment to hide unused acronyms
\vspace*{-120pt}
\printglossary[type=acronym]
\endgroup

\backmatter

\nocite{*}
\bibliography{biblio}
%\cleardoublepage
%\theindex %optional, use only if you have an index, must use
%\makeindex in the preamble


\end{document}

%%%%%%%%%%%%%%%%%%%%%%%%%%%%%%%%%%%%%%%%%%%%%%%%%%%%%%%%%%%%%%%%%%%%%%%%%%%%%%%%%%%%%%%%%%%%%%%%%%%%%%%%%%%%%%
